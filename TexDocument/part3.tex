\PassOptionsToPackage{unicode=true}{hyperref} % options for packages loaded elsewhere
\PassOptionsToPackage{hyphens}{url}
%
\documentclass[]{article}
\usepackage{lmodern}
\usepackage{amssymb,amsmath}

% settings
\usepackage{minted}
%



\usepackage{ifxetex,ifluatex}
\usepackage{fixltx2e} % provides \textsubscript
\ifnum 0\ifxetex 1\fi\ifluatex 1\fi=0 % if pdftex
  \usepackage[T1]{fontenc}
  \usepackage[utf8]{inputenc}
  \usepackage{textcomp} % provides euro and other symbols
\else % if luatex or xelatex
  \usepackage{unicode-math}
  \defaultfontfeatures{Ligatures=TeX,Scale=MatchLowercase}
    \setmainfont[]{Source Han Serif CN}
    \setsansfont[]{Source Han Sans CN}
    \setmonofont[Mapping=tex-ansi]{Source Code Pro}
  \ifxetex
    \usepackage{xeCJK}
    \setCJKmainfont[]{Source Han Serif CN}
  \fi
  \ifluatex
    \usepackage[]{luatexja-fontspec}
    \setmainjfont[]{Source Han Serif CN}
  \fi
\fi
% use upquote if available, for straight quotes in verbatim environments
\IfFileExists{upquote.sty}{\usepackage{upquote}}{}
% use microtype if available
\IfFileExists{microtype.sty}{%
\usepackage[]{microtype}
\UseMicrotypeSet[protrusion]{basicmath} % disable protrusion for tt fonts
}{}
\IfFileExists{parskip.sty}{%
\usepackage{parskip}
}{% else
\setlength{\parindent}{0pt}
\setlength{\parskip}{6pt plus 2pt minus 1pt}
}
\usepackage{hyperref}
\hypersetup{
            pdfborder={0 0 0},
            breaklinks=true}
\urlstyle{same}  % don't use monospace font for urls
\usepackage[margin=2cm]{geometry}
\setlength{\emergencystretch}{3em}  % prevent overfull lines
\providecommand{\tightlist}{%
  \setlength{\itemsep}{0pt}\setlength{\parskip}{0pt}}
\setcounter{secnumdepth}{0}
% Redefines (sub)paragraphs to behave more like sections
\ifx\paragraph\undefined\else
\let\oldparagraph\paragraph
\renewcommand{\paragraph}[1]{\oldparagraph{#1}\mbox{}}
\fi
\ifx\subparagraph\undefined\else
\let\oldsubparagraph\subparagraph
\renewcommand{\subparagraph}[1]{\oldsubparagraph{#1}\mbox{}}
\fi


% set default figure placement to htbp
\makeatletter
\def\fps@figure{htbp}
\makeatother

\usepackage{minted}



\date{}

\title{\vspace{50mm} \huge Standard Code Library \\[20pt]}
\author{HeartFireY, eroengine, yezzz \\[10pt] Jiangxi Normal University}
\date{\today}


\begin{document}

\begin{titlepage}

\maketitle

\end{titlepage}

\newpage

\renewcommand\labelitemi{$\bullet$}

{
\setcounter{tocdepth}{3}
\tableofcontents
\newpage
}








\hypertarget{section.5-ux5b57ux7b26ux4e32}{%
\section{Section.5 字符串}\label{section.5-ux5b57ux7b26ux4e32}}

\hypertarget{kmp}{%
\subsection{KMP}\label{kmp}}

\begin{itemize}
\tightlist
\item
  前缀函数(每一个前缀的最长 border)
\end{itemize}

\begin{minted}[fontsize=\footnotesize,breaklines,linenos]{cpp}
int nxt[N];

void get_pi(int nxt[], string s, int n){
    int j = nxt[0] = 0;
    for(int i = 2; i <= n; i++){
        while(j && s[i] != s[j + 1]) j = nxt[j];
        nxt[i] = j += s[i] == s[j + 1];
    }
}

void kmp(string s, string p, int lens, int lenp){
    for(int i = 1, j = 0; i <= lens; i++){
        while(j > 0 && s[i] != p[j + 1]) j = nxt[j];
        j += s[i] == p[j + 1];
        if(j == lenp){
            cout << i - lenp + 1 << endl;
            j = nxt[j];
        }
    }
}
\end{minted}

\begin{itemize}
\tightlist
\item
  Z 函数(每一个后缀和该字符串的 LCP 长度)
\end{itemize}

\begin{minted}[fontsize=\footnotesize,breaklines,linenos]{cpp}
void get_z(int a[], char s[], int n) {
    int l = 0, r = 0; a[0] = n;
    FOR (i, 1, n) {
        a[i] = i > r ? 0 : min(r - i + 1, a[i - l]);
        while (i + a[i] < n && s[a[i]] == s[i + a[i]]) ++a[i];
        if (i + a[i] - 1 > r) { l = i; r = i + a[i] - 1; }
    }
}
\end{minted}

\hypertarget{trie}{%
\subsection{Trie}\label{trie}}

\begin{minted}[fontsize=\footnotesize,breaklines,linenos]{cpp}
namespace trie {
    int t[N][26], sz, ed[N];
    void init() { sz = 2; memset(ed, 0, sizeof ed); }
    int _new() { memset(t[sz], 0, sizeof t[sz]); return sz++; }
    void ins(char* s, int p) {
        int u = 1;
        FOR (i, 0, strlen(s)) {
            int c = s[i] - 'a';
            if (!t[u][c]) t[u][c] = _new();
            u = t[u][c];
        }
        ed[u] = p;
    }
}
\end{minted}

\hypertarget{ac-ux81eaux52a8ux673a}{%
\subsection{AC 自动机}\label{ac-ux81eaux52a8ux673a}}

\begin{minted}[fontsize=\footnotesize,breaklines,linenos]{cpp}
const int N = 1e6 + 10, M = 26, MOD = 1e9 + 7;

#define mp(x) (x - 'a')

namespace ACA{
    int ch[N][M], fail[N], ed[N];
    int sz;
    void init() {
        sz = 1;
        memset(ch[0], 0, sizeof(ch));
    }
    
    void insert(const string &s, int id) {
        int n = s.size(), u = 0, c;
        for(int i = 0; i < n; i++) {
            c = mp(s[i]);
            if(!ch[u][c]) {
                memset(ch[sz], 0, sizeof(ch[sz]));
                ch[u][c] = sz++;
            }
            u = ch[u][c];
        }
        ed[u] = id;
    }

    void build() {
        queue<int> Q;
        fail[0] = 0;
        for(int c = 0, u; c < M; c++) {
            u = ch[0][c];
            if(u) { Q.emplace(u); fail[u] = 0;  }
        }
        while(!Q.empty()) {
            int r = Q.front(); Q.pop();
            for(int c = 0, u; c < M; c++) {
                u = ch[r][c];
                if(!u) {
                    ch[r][c] = ch[fail[r]][c];
                    continue;
                }
                fail[u] = ch[fail[r]][c];
                Q.emplace(u);
            }
        }
    }
}

\end{minted}

\hypertarget{ux540eux7f00ux81eaux52a8ux673a}{%
\subsection{后缀自动机}\label{ux540eux7f00ux81eaux52a8ux673a}}

\begin{itemize}
\tightlist
\item
  广义后缀自动机如果直接使用以下代码的话会产生一些冗余状态(置 last 为
  1),所以要用拓扑排序。用 len 基数排序不能。
\item
  字符集大的话要使用 \texttt{map}。
\item
  树上 dp 时注意边界(root 和 null)。
\item
  rsort 中的数组 a 是拓扑序 {[}1, sz)
\end{itemize}

\begin{minted}[fontsize=\footnotesize,breaklines,linenos]{cpp}
struct SAM{
    int ch[N << 1][26], fa[N << 1], len[N << 1], vis[N << 1];
    int last, tot;
    SAM(): last(1), tot(1) {}
    inline void extend(int x){ //*单字符扩展
        int p = last, np = last = ++tot;
        len[np] = len[p] + 1, vis[np] = 1;
        for(; p && !ch[p][x]; p = fa[p]) ch[p][x] = np;
        if(!p) fa[np] = 1;
        else{
            int q = ch[p][x];
            if(len[q] == len[p] + 1) fa[np] = q;
            else {
                int nq = ++tot;
                for(int i = 0; i < 26; i++) ch[nq][i] = ch[q][i]; //for(int i = 0; i < 26; i++) ch[nq][i] = ch[q][i];
                fa[nq] = fa[q], fa[np] = fa[q] = nq, len[nq] = len[p] + 1;
                for(; ch[p][x] == q; p = fa[p]) ch[p][x] = nq;
            }
        }
    }
}sam;
\end{minted}

\begin{itemize}
\tightlist
\item
  最长公共子串
\end{itemize}

\begin{minted}[fontsize=\footnotesize,breaklines,linenos]{cpp}
//*最长公共子串
string lcs(const string &T) {
    int v = 0, l = 0, best = 0, bestpos = 0;
    for (int i = 0; i < T.size(); i++) {
        while (v && !sam.ch[v][T[i] - 'a']) {
            v = sam.fa[v];
            l = sam.len[v];
        }   
        if (sam.ch[v][T[i] - 'a']) {
            v = sam.ch[v][T[i] - 'a'];
            l++;
        }
        if (l > best) {
            best = l;
            bestpos = i;
        }
    }
    return T.substr(bestpos - best + 1, best);
}
\end{minted}

\begin{itemize}
\tightlist
\item
  真·广义后缀自动机
\end{itemize}

\begin{minted}[fontsize=\footnotesize,breaklines,linenos]{cpp}
int t[M][26], len[M] = {-1}, fa[M], sz = 2, last = 1;
LL cnt[M][2];
void ins(int ch, int id) {
    int p = last, np = 0, nq = 0, q = -1;
    if (!t[p][ch]) {
        np = sz++;
        len[np] = len[p] + 1;
        for (; p && !t[p][ch]; p = fa[p]) t[p][ch] = np;
    }
    if (!p) fa[np] = 1;
    else {
        q = t[p][ch];
        if (len[p] + 1 == len[q]) fa[np] = q;
        else {
            nq = sz++; len[nq] = len[p] + 1;
            memcpy(t[nq], t[q], sizeof t[0]);
            fa[nq] = fa[q];
            fa[np] = fa[q] = nq;
            for (; t[p][ch] == q; p = fa[p]) t[p][ch] = nq;
        }
    }
    last = np ? np : nq ? nq : q;
    cnt[last][id] = 1;
}
\end{minted}

\begin{itemize}
\tightlist
\item
  按字典序建立后缀树 注意逆序插入
\item
  rsort2 里的 a 不是拓扑序,需要拓扑序就去树上做
\end{itemize}

\begin{minted}[fontsize=\footnotesize,breaklines,linenos]{cpp}
void ins(int ch, int pp) {
    int p = last, np = last = sz++;
    len[np] = len[p] + 1; one[np] = pos[np] = pp;
    for (; p && !t[p][ch]; p = fa[p]) t[p][ch] = np;
    if (!p) { fa[np] = 1; return; }
    int q = t[p][ch];
    if (len[q] == len[p] + 1) fa[np] = q;
    else {
        int nq = sz++; len[nq] = len[p] + 1; one[nq] = one[q];
        memcpy(t[nq], t[q], sizeof t[0]);
        fa[nq] = fa[q];
        fa[q] = fa[np] = nq;
        for (; p && t[p][ch] == q; p = fa[p]) t[p][ch] = nq;
    }
}

int up[M], c[256] = {2}, a[M];
void rsort2() {
    FOR (i, 1, 256) c[i] = 0;
    FOR (i, 2, sz) up[i] = s[one[i] + len[fa[i]]];
    FOR (i, 2, sz) c[up[i]]++;
    FOR (i, 1, 256) c[i] += c[i - 1];
    FOR (i, 2, sz) a[--c[up[i]]] = i;
    FOR (i, 2, sz) G[fa[a[i]]].push_back(a[i]);
}
\end{minted}

\begin{itemize}
\tightlist
\item
  广义后缀自动机建后缀树,必须反向插入
\end{itemize}

\begin{minted}[fontsize=\footnotesize,breaklines,linenos]{cpp}
int t[M][26], len[M] = {0}, fa[M], sz = 2, last = 1;
char* one[M];
void ins(int ch, char* pp) {
    int p = last, np = 0, nq = 0, q = -1;
    if (!t[p][ch]) {
        np = sz++; one[np] = pp;
        len[np] = len[p] + 1;
        for (; p && !t[p][ch]; p = fa[p]) t[p][ch] = np;
    }
    if (!p) fa[np] = 1;
    else {
        q = t[p][ch];
        if (len[p] + 1 == len[q]) fa[np] = q;
        else {
            nq = sz++; len[nq] = len[p] + 1; one[nq] = one[q];
            memcpy(t[nq], t[q], sizeof t[0]);
            fa[nq] = fa[q];
            fa[np] = fa[q] = nq;
            for (; t[p][ch] == q; p = fa[p]) t[p][ch] = nq;
        }
    }
    last = np ? np : nq ? nq : q;
}
int up[M], c[256] = {2}, aa[M];
vector<int> G[M];
void rsort() {
    FOR (i, 1, 256) c[i] = 0;
    FOR (i, 2, sz) up[i] = *(one[i] + len[fa[i]]);
    FOR (i, 2, sz) c[up[i]]++;
    FOR (i, 1, 256) c[i] += c[i - 1];
    FOR (i, 2, sz) aa[--c[up[i]]] = i;
    FOR (i, 2, sz) G[fa[aa[i]]].push_back(aa[i]);
}
\end{minted}

\begin{itemize}
\tightlist
\item
  匹配
\end{itemize}

\begin{minted}[fontsize=\footnotesize,breaklines,linenos]{cpp}
int u = 1, l = 0;
FOR (i, 0, strlen(s)) {
    int ch = s[i] - 'a';
    while (u && !t[u][ch]) { u = fa[u]; l = len[u]; }
    ++l; u = t[u][ch];
    if (!u) u = 1;
    if (l) // do something...
}
\end{minted}

\begin{itemize}
\tightlist
\item
  获取子串状态
\end{itemize}

\begin{minted}[fontsize=\footnotesize,breaklines,linenos]{cpp}
int get_state(int l, int r) {
    int u = rpos[r], s = r - l + 1;
    FORD (i, SP - 1, -1) if (len[pa[u][i]] >= s) u = pa[u][i];
    return u;
}
\end{minted}

\begin{itemize}
\item
  维护区间本质不同字串数目

  \begin{itemize}
  \item
    给你一个长度为\(n\)的字符串\(s\),\(m\)次询问,第\(i\)次询问\(s\)上的一个区间\([l_i,r_i]\)上有多少个本质不同的子串
  \item
    将每个本质不同的字符串视为一个连续的区间\([l,r]\),我们只需要维护左端点最后一次出现的位置即可

    因为需要知道每个子串最后一次出现的位置,所以我们选择对字符串构造后缀自动机,对于某个右端点\$
    r \(来说,对应到 SAM 上,从最长的那个前缀\) {[} 1 , r {]}
    \(对应的节点开始,沿着\) parent \(树到根节点的这条路径上,都是以\) r
    \(为右端点的子串,设\) pre{[} i {]} \(为节点\) i
    \$上一次出现时的右端点位置,我们只需要暴跳
    \(father\),每次将之前出现过的位置,在线段树上区间更新成 \$-1
    \((因为对于每个节点来说,代表的都是一段连续的后缀,所以可以用到线段树的区间更新),最后再将\)
    {[} 1 , r {]} , {[} 2 , r {]} \ldots{} {[} r , r {]} \(这\) r
    \(个后缀所代表的子串的左端点,即\) {[} 1 , r {]} \$在线段树上 \$+1
    \$就好了

    不过问题是,这样暴跳\$ father
    \$的时间复杂度是不正确的,考虑优化,因为每次选择一条链,自下而上去更新,其本质就是:

    \begin{enumerate}
    \def\labelenumi{\arabic{enumi}.}
    \tightlist
    \item
      令这条链每个节点相应的位置在线段树上区间更新
    \item
      令每个节点都被端点\$ r \(覆盖(也就是将\) pre \(都标记为\) r\$)
    \end{enumerate}

    这个过程其实就是\$ LCT \(中的\) access
    \(函数的操作,这样每次更新至多需要更新\) logn \(个\)
    splay\(,套上线段树就是\) log\^{}2n\(,均摊下来就是\) nlog\^{}2n
    \$了。
  \end{itemize}
\end{itemize}

\begin{minted}[fontsize=\footnotesize,breaklines,linenos]{cpp}
#include <bits/stdc++.h>
#pragma gcc optimize("O2")
#pragma g++ optimize("O2")
#define int long long
#define endl '\n'
using namespace std;

const int N = 2e5 + 10, MOD = 1e9 + 7;

int n = 0;

namespace DS{
    namespace SAM{
        int ch[N << 1][26], fa[N << 1], len[N << 1], vis[N << 1], pos[N << 1];
        int last, tot;
        inline void extend(int x){ 
            int p = last, np = last = ++tot;
            len[np] = len[p] + 1, vis[np] = 1;
            for(; p && !ch[p][x]; p = fa[p]) ch[p][x] = np;
            if(!p) fa[np] = 1;
            else{
                int q = ch[p][x];
                if(len[q] == len[p] + 1) fa[np] = q;
                else {
                    int nq = ++tot;
                    for(int i = 0; i < 26; i++) ch[nq][i] = ch[q][i];
                    fa[nq] = fa[q], fa[np] = fa[q] = nq, len[nq] = len[p] + 1;
                    for(; ch[p][x] == q; p = fa[p]) ch[p][x] = nq;
                }
            }
        }   

        void build(string s) {
            last = tot = 1;
            int len = s.size();
            s = '@' + s;
            for(int i = 1; i <= len; i++) extend(s[i] - 'a'), pos[i] = last;
        }
    }

    namespace SegTree{
        #define ls rt << 1
        #define rs rt << 1 | 1
        #define lson ls, l, mid
        #define rson rs, mid + 1, r
        int tree[N << 2], lazy[N << 2];
      
        inline void push_up(int rt) { tree[rt] = tree[ls] + tree[rs]; }

        inline void push(int rt, int val, int c) { tree[rt] += val * c, lazy[rt] += val; }

        inline void push_down(int rt, int c) {
            if(lazy[rt]) {
                push(ls, lazy[rt], (c - (c >> 1)));
                push(rs, lazy[rt], (c >> 1));
                lazy[rt] = 0;
            }
        }

        void build(int rt, int l, int r){
            tree[rt] = lazy[rt] = 0;
            if(l == r) return;
            int mid = l + r >> 1;
            build(lson), build(rson);
        }

        void update(int rt, int l, int r, int L, int R, int val) {
            if(l >= L && r <= R) return push(rt, val, r - l + 1);
            push_down(rt, r - l + 1);
            int mid = l + r >> 1;
            if(mid >= L) update(lson, L, R, val);
            if(mid < R) update(rson, L, R, val);
            push_up(rt);
        }

        int query(int rt, int l, int r, int L, int R) {
            if(l >= L && r <= R) return tree[rt];
            push_down(rt, r - l + 1);
            int mid = l + r >> 1, sum = 0;
            if(mid >= L) sum += query(lson, L, R);
            if(mid < R) sum += query(rson, L, R);
            return sum;
        }
        #undef ls
        #undef rs
        #undef lson
        #undef rson
    }

    namespace LCT{
        #define ls ch[x][0]
        #define rs ch[x][1]

        struct Info{
            int len, minn, pre, tag_chg;
        }tree[N];

        int ch[N][2], f[N], tag[N];

        inline void push_up(int x) { tree[x].minn = min({tree[x].len, tree[ls].minn, tree[rs].minn}); }

        inline void push(int x) { swap(ls, rs), tag[x] ^= 1; }

        inline void push_chg(int x, int v) { tree[x].pre = tree[x].tag_chg = v; }

        inline void push_down(int x) {
            if(tag[x]) {
                if(ls) push(ls);
                if(rs) push(rs);
                tag[x] = 0;
            }
            if(tree[x].tag_chg) {
                if(ls) push_chg(ls, tree[x].tag_chg);
                if(rs) push_chg(rs, tree[x].tag_chg);
                tree[x].tag_chg = 0;
            }
        }

        #define get(x) (ch[f[x]][1] == x)                      
        #define isRoot(x) (ch[f[x]][0] != x && ch[f[x]][1] != x) 

        inline void rotate(int x) {
            int y = f[x], z = f[y], k = get(x);
            if(!isRoot(y)) ch[z][ch[z][1] == y] = x;
            ch[y][k] = ch[x][!k], f[ch[x][!k]] = y;
            ch[x][!k] = y, f[y] = x, f[x] = z;
            push_up(y); push_up(x);
        }

        inline void update(int x) { 
            if(!isRoot(x)) update(f[x]);
            push_down(x);
        }

        inline void splay(int x) {
            update(x);
            for(int fa = f[x]; !isRoot(x); rotate(x), fa = f[x]){
                if(!isRoot(fa)) rotate(get(fa) == get(x) ? fa : x);
            }
            push_up(x);
        }

        int access(int x, int pos) {
            int p;
            for(p = 0; x; x = f[p = x]){
                splay(x), ch[x][1] = p, push_up(x);
                if(tree[x].pre) {
                    int upl = tree[x].pre - SAM::len[x] + 1;
                    int upr = tree[x].pre - tree[x].minn + 1;
                    // cout << "LCT Operation SegTree -> Part(" << upl << ", " << upr << "), add value -1\n";
                    SegTree::update(1, 1, n, upl, upr, -1);
                }
            }
            splay(p);
            push_chg(p, pos);
            SegTree::update(1, 1, n, 1, pos, 1);
            // cout << endl;
            return p;
        }

        void build() {
            tree[0].minn = 1e18;
            for(int i = 1; i <= SAM::tot; i++) {
                f[i] = SAM::fa[i];
                tree[i].len = tree[i].minn = SAM::len[SAM::fa[i]] + 1;
                tree[i].pre = tree[i].tag_chg = ch[i][0] = ch[i][1] = 0;
            }
        }
        #undef ls
        #undef rs
    }
}

struct query{ int l, id; };
vector<query> qr[N];
int ans[N];

inline void solve(){
    string s; cin >> s, n = s.size();
    DS::SAM::build(s);
    DS::SegTree::build(1, 1, n);
    DS::LCT::build();
    int m = 0; cin >> m;
    for(int i = 1; i <= m; i++) {
        int l, r; cin >> l >> r;
        qr[r].emplace_back(query{l, i});      
    }
    for(int i = 1; i <= n; i++) {
        DS::LCT::access(DS::SAM::pos[i], i);
        for(auto &[l, id] : qr[i]) ans[id] = DS::SegTree::query(1, 1, n, l, i);
    }
    for(int i = 1; i <= m; i++) cout << ans[i] << endl;
}

signed main(){
    ios_base::sync_with_stdio(false), cin.tie(0);
    cout << fixed << setprecision(12);
    int t = 1; // cin >> t;
    while(t--) solve();
    return 0;
}
\end{minted}

\hypertarget{ux56deux6587ux81eaux52a8ux673a}{%
\subsection{回文自动机}\label{ux56deux6587ux81eaux52a8ux673a}}

\begin{itemize}
\tightlist
\item
  num 是该结点表示的前缀的回文后缀个数
\item
  cnt 是该结点表示的回文串在原串中的出现次数(使用前需要向父亲更新)
\end{itemize}

\begin{minted}[fontsize=\footnotesize,breaklines,linenos]{cpp}
namespace pam {
    int t[N][26], fa[N], len[N], rs[N], cnt[N], num[N];
    int sz, n, last;
    int _new(int l) {
        len[sz] = l; cnt[sz] = num[sz] = 0;
        return sz++;
    }
    void init() {
        memset(t, 0, sz * sizeof t[0]);
        rs[n = sz = 0] = -1;
        last = _new(0);
        fa[last] = _new(-1);
    }
    int get_fa(int x) {
        while (rs[n - 1 - len[x]] != rs[n]) x = fa[x];
        return x;
    }
    void ins(int ch) {
        rs[++n] = ch;
        int p = get_fa(last);
        if (!t[p][ch]) {
            int np = _new(len[p] + 2);
            num[np] = num[fa[np] = t[get_fa(fa[p])][ch]] + 1;
            t[p][ch] = np;
        }
        ++cnt[last = t[p][ch]];
    }
}
\end{minted}

\hypertarget{manacher}{%
\subsection{Manacher}\label{manacher}}

\(P[i]\)
的计算方法:假设\(j\)是\(i\)关于\(C\)的镜像点,\(P[j]\)已经求解完毕。

\begin{itemize}
\tightlist
\item
  若\(i \geq R\),由于\(R\)右侧的字符都没有检查过,因此只能初始化\(P[i] = 1\)然后暴力中心扩展
\item
  若\(i < R\),分两种情况:

  \begin{enumerate}
  \def\labelenumi{\arabic{enumi}.}
  \tightlist
  \item
    \(j\)的回文串被\(C\)的回文串包含,即\(j\)的回文串左端点比\(C\)回文串的左端点大,按照镜像原理,镜像\(i\)的回文不会超过\(C\)的右端点\(R\),因此根据\((i + j) / 2 = C\)得\(j = 2C - i\),故\(P[i] = P[j] = P[2C - i]\)。然后继续用暴力中心扩展法完成\(P[i]\)的计算。
  \item
    \(j\)的回文串不被\(C\)的回文串包含,即\(j\)的回文串左端点比\(C\)回文串的左端点小。此时\(i\)回文串的右端点比\(R\)大,但是由于\(R\)右边的字符还没有检查过,只能先让\(P[i]\)被限制在\(R\)之内,有\(P[i] = w =R - i = C + P[i] - i\),然后继续用暴力中心扩展法完成\(P[i]\)的计算。
  \end{enumerate}
\end{itemize}

\begin{minted}[fontsize=\footnotesize,breaklines,linenos]{cpp}
int n, p[N << 1];       // p[i]: 以s[i]为中心的回文串半径
char a[N], s[N << 1];   // a为原始串, s为修改后的串

void change() {
    n = strlen(a);
    int k = 0; s[k++] = '$', s[k++] = '#';
    for(int i = 0; i < n; i++) s[k++] = a[i], s[k++] = '#';
    s[k++] = '&', n = k;
}

void manacher() {
    int R = 0, C = 0;
    for(int i = 1; i < n; i++) {
        if(i < R) p[i] = min(p[(C << 1) - i], p[C] + C - i);    // 1.合并处理两种情况
        else p[i] = 1;                                          //   
        while(s[i + p[i]] == s[i - p[i]]) p[i]++;               // 2.暴力中心扩展
        if(p[i] + i > R) R = p[i] + i, C = i;                   // 3.更新最大的R
    }   
}

inline void solve() {
    cin >> a;
    change(), manacher();
    int ans = 1;
    for(int i = 0; i < n; i++) ans = max(ans, p[i]);
    cout << ans - 1 << endl;
}
\end{minted}

\hypertarget{ux54c8ux5e0c}{%
\subsection{哈希}\label{ux54c8ux5e0c}}

内置了自动双哈希开关(小心 TLE)。

\begin{minted}[fontsize=\footnotesize,breaklines,linenos]{cpp}
#include <bits/stdc++.h>
using namespace std;

#define ENABLE_DOUBLE_HASH

typedef long long LL;
typedef unsigned long long ULL;

const int x = 135;
const int N = 4e5 + 10;
const int p1 = 1e9 + 7, p2 = 1e9 + 9;
ULL xp1[N], xp2[N], xp[N];

void init_xp() {
    xp1[0] = xp2[0] = xp[0] = 1;
    for (int i = 1; i < N; ++i) {
        xp1[i] = xp1[i - 1] * x % p1;
        xp2[i] = xp2[i - 1] * x % p2;
        xp[i] = xp[i - 1] * x;
    }
}

struct String {
    char s[N];
    int length, subsize;
    bool sorted;
    ULL h[N], hl[N];

    ULL hash() {
        length = strlen(s);
        ULL res1 = 0, res2 = 0;
        h[length] = 0;  // ATTENTION!
        for (int j = length - 1; j >= 0; --j) {
        #ifdef ENABLE_DOUBLE_HASH
            res1 = (res1 * x + s[j]) % p1;
            res2 = (res2 * x + s[j]) % p2;
            h[j] = (res1 << 32) | res2;
        #else
            res1 = res1 * x + s[j];
            h[j] = res1;
        #endif
            // printf("%llu\n", h[j]);
        }
        return h[0];
    }

    // 获取子串哈希,左闭右开区间
    ULL get_substring_hash(int left, int right) const {
        int len = right - left;
    #ifdef ENABLE_DOUBLE_HASH
        // get hash of s[left...right-1]
        unsigned int mask32 = ~(0u);
        ULL left1 = h[left] >> 32, right1 = h[right] >> 32;
        ULL left2 = h[left] & mask32, right2 = h[right] & mask32;
        return (((left1 - right1 * xp1[len] % p1 + p1) % p1) << 32) |
               (((left2 - right2 * xp2[len] % p2 + p2) % p2));
    #else
        return h[left] - h[right] * xp[len];
    #endif
    }

    void get_all_subs_hash(int sublen) {
        subsize = length - sublen + 1;
        for (int i = 0; i < subsize; ++i)
            hl[i] = get_substring_hash(i, i + sublen);
        sorted = 0;
    }

    void sort_substring_hash() {
        sort(hl, hl + subsize);
        sorted = 1;
    }

    bool match(ULL key) const {
        if (!sorted) assert (0);
        if (!subsize) return false;
        return binary_search(hl, hl + subsize, key);
    }

    void init(const char *t) {
        length = strlen(t);
        strcpy(s, t);
    }
};

int LCP(const String &a, const String &b, int ai, int bi) {
    // Find LCP of a[ai...] and b[bi...]
    int l = 0, r = min(a.length - ai, b.length - bi);
    while (l < r) {
        int mid = (l + r + 1) / 2;
        if (a.get_substring_hash(ai, ai + mid) == b.get_substring_hash(bi, bi + mid))
            l = mid;
        else r = mid - 1;
    }
    return l;
}

int check(int ans) {
    if (T.length < ans) return 1;
    T.get_all_subs_hash(ans); T.sort_substring_hash();
    for (int i = 0; i < S.length - ans + 1; ++i)
        if (!T.match(S.get_substring_hash(i, i + ans)))
            return 1;
    return 0;
}

int main() {
    init_xp();  // DON'T FORGET TO DO THIS!

    for (int tt = 1; tt <= kases; ++tt) {
        scanf("%d", &n); scanf("%s", str);
        S.init(str);
        S.hash(); T.hash();
    }
}
\end{minted}

二维哈希

\begin{minted}[fontsize=\footnotesize,breaklines,linenos]{cpp}
struct Hash2D { // 1-index
    static const LL px = 131, py = 233, MOD = 998244353;
    static LL pwx[N], pwy[N];
    int a[N][N];
    LL hv[N][N];
    static void init_xp() {
        pwx[0] = pwy[0] = 1;
        FOR (i, 1, N) {
            pwx[i] = pwx[i - 1] * px % MOD;
            pwy[i] = pwy[i - 1] * py % MOD;
        }
    }
    void init_hash(int n, int m) {
        FOR (i, 1, n + 1) {
            LL s = 0;
            FOR (j, 1, m + 1) {
                s = (s * py + a[i][j]) % MOD;
                hv[i][j] = (hv[i - 1][j] * px + s) % MOD;
            }
        }
    }
    LL h(int x, int y, int dx, int dy) {
        --x; --y;
        LL ret = hv[x + dx][y + dy] + hv[x][y] * pwx[dx] % MOD * pwy[dy]
                 - hv[x][y + dy] * pwx[dx] - hv[x + dx][y] * pwy[dy];
        return (ret % MOD + MOD) % MOD;
    }
} ha, hb;
LL Hash2D::pwx[N], Hash2D::pwy[N];
\end{minted}

\hypertarget{ux540eux7f00ux6570ux7ec4}{%
\subsection{后缀数组}\label{ux540eux7f00ux6570ux7ec4}}

\hypertarget{onlognux6784ux9020}{%
\subsubsection{O(nlogn)构造}\label{onlognux6784ux9020}}

构造时间:\(O(L \log L)\);查询时间 \(O(\log L)\)。\texttt{suffix}
数组是排好序的后缀下标, \texttt{suffix} 的反数组是后缀数组。

\begin{minted}[fontsize=\footnotesize,breaklines,linenos]{cpp}
const int N = 1e6 + 10;

namespace SA {
    int sa[N], tax[N], pool1[N], pool2[N], height[N];
    int *rnk = pool1, *tp = pool2;
    void sort(int n, int m) {
        for(int i = 0; i <= m; i++) tax[i] = 0;
        for(int i = 1; i <= n; i++) tax[rnk[i]]++;
        for(int i = 1; i <= m; i++) tax[i] += tax[i - 1];
        for(int i = n; i >= 1; i--) sa[tax[rnk[tp[i]]]--] = tp[i];
    }

    void build(string str, int n, int m) {
        for(int i = 1; i <= n; i++) rnk[i] = str[i] - '0' + 1, tp[i] = i;
        sort(n, m);
        for(int w = 1, p = 0; p < n; w <<= 1, m = p) {
            p = 0;
            for(int i = 1; i <= w; i++) tp[++p] = n - w + i;
            for(int i = 1; i <= n; i++) if(sa[i] > w) tp[++p] = sa[i] - w;
            sort(n, m);
            swap(tp, rnk);
            rnk[sa[1]] = p = 1;
            for(int i = 2; i <= n; i++) rnk[sa[i]] = (tp[sa[i - 1]] == tp[sa[i]] && tp[sa[i - 1] + w] == tp[sa[i] + w]) ? p : ++p;
        }
        int k = 0;
        for(int i = 1; i <= n; i++) {
            if(rnk[i] == 1) continue;
            if(k) --k;
            for(int j = sa[rnk[i] - 1]; str[i + k] == str[j + k]; ++k);
            height[rnk[i]] = k;
        } // Get the height array (height[i] = lcp(sa[i], sa[i - 1]))
    } 

    void reset() {
        memset(sa, 0, sizeof(sa));
        memset(rnk, 0, sizeof(pool1));
        memset(tp, 0, sizeof(pool2));
        memset(tax, 0, sizeof(tax));
        rnk = pool1, tp = pool2;
    }
} // Suffix Array (init pos = 1)

using SA::sa;

inline void solve() {
    string s; cin >> s;
    int len = s.size();
    SA::bulid('@' + s, len, 75);
    for(int i = 1; i <= len; i++) cout << sa[i] << " \n"[i == len];
}
\end{minted}

\hypertarget{heightux6570ux7ec4ux7684ux5e94ux7528}{%
\subsubsection{Height数组的应用}\label{heightux6570ux7ec4ux7684ux5e94ux7528}}

\hypertarget{ux4e24ux4e2aux4e32ux7684ux6700ux957fux516cux5171ux524dux7f00}{%
\paragraph{(1).两个串的最长公共前缀}\label{ux4e24ux4e2aux4e32ux7684ux6700ux957fux516cux5171ux524dux7f00}}

\(lcp(sa[i],sa[j])=\min\{height[i+1..j]\}\)

如果 \(height\)
一直大于某个数,前这么多位就一直没变过;反之,由于后缀已经排好序了,不可能变了之后变回来。

\begin{minted}[fontsize=\footnotesize,breaklines,linenos]{cpp}
namespace SA {...}


\end{minted}

\hypertarget{ux6bd4ux8f83ux4e00ux4e2aux5b57ux7b26ux4e32ux7684ux4e24ux4e2aux5b50ux4e32ux7684ux5927ux5c0fux5173ux7cfb}{%
\paragraph{(2).比较一个字符串的两个子串的大小关系}\label{ux6bd4ux8f83ux4e00ux4e2aux5b57ux7b26ux4e32ux7684ux4e24ux4e2aux5b50ux4e32ux7684ux5927ux5c0fux5173ux7cfb}}

假设需要比较的是 \(A=S[a..b]\) 和 \(B=S[c..d]\) 的大小关系。

若 \(lcp(a, c)\ge\min(|A|, |B|)\),\(A<B\iff |A|<|B|\)。

否则,\(A<B\iff rk[a]< rk[c]\)。

\hypertarget{ux4e0dux540cux5b50ux4e32ux7684ux6570ux76ee}{%
\paragraph{(3).不同子串的数目}\label{ux4e0dux540cux5b50ux4e32ux7684ux6570ux76ee}}

子串就是后缀的前缀,所以可以枚举每个后缀,计算前缀总数,再减掉重复。

``前缀总数''其实就是子串个数,为 \(n(n+1)/2\)。

如果按后缀排序的顺序枚举后缀,每次新增的子串就是除了与上一个后缀的 LCP
剩下的前缀。这些前缀一定是新增的,否则会破坏
\(lcp(sa[i],sa[j])=\min\{height[i+1..j]\}\)
的性质。只有这些前缀是新增的,因为 LCP 部分在枚举上一个前缀时计算过了。

所以答案为:\(\frac{n(n+1)}{2}-\sum\limits_{i=2}^nheight[i]\)

\hypertarget{onux6784ux9020}{%
\subsubsection{O(n)构造}\label{onux6784ux9020}}

\begin{itemize}
\tightlist
\item
  SA-IS
\item
  仅在后缀自动机被卡内存或者卡常且需要 O(1) LCA
  的情况下使用(比赛中敲这个我觉得不行)
\item
  UOJ 35
\end{itemize}

\begin{minted}[fontsize=\footnotesize,breaklines,linenos]{cpp}
// rk [0..n-1] -> [1..n], sa/ht [1..n]
// s[i] > 0 && s[n] = 0
// b: normally as bucket
// c: normally as bucket1
// d: normally as bucket2
// f: normally as cntbuf

template<size_t size>
struct SuffixArray {
    bool t[size << 1];
    int b[size], c[size];
    int sa[size], rk[size], ht[size];
    inline bool isLMS(const int i, const bool *t) { return i > 0 && t[i] && !t[i - 1]; }
    template<class T>
    inline void inducedSort(T s, int *sa, const int n, const int M, const int bs,
                            bool *t, int *b, int *f, int *p) {
        fill(b, b + M, 0); fill(sa, sa + n, -1);
        FOR (i, 0, n) b[s[i]]++;
        f[0] = b[0];
        FOR (i, 1, M) f[i] = f[i - 1] + b[i];
        FORD (i, bs - 1, -1) sa[--f[s[p[i]]]] = p[i];
        FOR (i, 1, M) f[i] = f[i - 1] + b[i - 1];
        FOR (i, 0, n) if (sa[i] > 0 && !t[sa[i] - 1]) sa[f[s[sa[i] - 1]]++] = sa[i] - 1;
        f[0] = b[0];
        FOR (i, 1, M) f[i] = f[i - 1] + b[i];
        FORD (i, n - 1, -1) if (sa[i] > 0 && t[sa[i] - 1]) sa[--f[s[sa[i] - 1]]] = sa[i] - 1;
    }
    template<class T>
    inline void sais(T s, int *sa, int n, bool *t, int *b, int *c, int M) {
        int i, j, bs = 0, cnt = 0, p = -1, x, *r = b + M;
        t[n - 1] = 1;
        FORD (i, n - 2, -1) t[i] = s[i] < s[i + 1] || (s[i] == s[i + 1] && t[i + 1]);
        FOR (i, 1, n) if (t[i] && !t[i - 1]) c[bs++] = i;
        inducedSort(s, sa, n, M, bs, t, b, r, c);
        for (i = bs = 0; i < n; i++) if (isLMS(sa[i], t)) sa[bs++] = sa[i];
        FOR (i, bs, n) sa[i] = -1;
        FOR (i, 0, bs) {
            x = sa[i];
            for (j = 0; j < n; j++) {
                if (p == -1 || s[x + j] != s[p + j] || t[x + j] != t[p + j]) { cnt++, p = x; break; }
                else if (j > 0 && (isLMS(x + j, t) || isLMS(p + j, t))) break;
            }
            x = (~x & 1 ? x >> 1 : x - 1 >> 1), sa[bs + x] = cnt - 1;
        }
        for (i = j = n - 1; i >= bs; i--) if (sa[i] >= 0) sa[j--] = sa[i];
        int *s1 = sa + n - bs, *d = c + bs;
        if (cnt < bs) sais(s1, sa, bs, t + n, b, c + bs, cnt);
        else FOR (i, 0, bs) sa[s1[i]] = i;
        FOR (i, 0, bs) d[i] = c[sa[i]];
        inducedSort(s, sa, n, M, bs, t, b, r, d);
    }
    template<typename T>
    inline void getHeight(T s, const int n, const int *sa) {
        for (int i = 0, k = 0; i < n; i++) {
            if (rk[i] == 0) k = 0;
            else {
                if (k > 0) k--;
                int j = sa[rk[i] - 1];
                while (i + k < n && j + k < n && s[i + k] == s[j + k]) k++;
            }
            ht[rk[i]] = k;
        }
    }
    template<class T>
    inline void init(T s, int n, int M) {
        sais(s, sa, ++n, t, b, c, M);
        for (int i = 1; i < n; i++) rk[sa[i]] = i;
        getHeight(s, n, sa);
    }
};

const int N = 2E5 + 100;
SuffixArray<N> sa;

int main() {
    string s; cin >> s; int n = s.length();
    sa.init(s, n, 128);
    FOR (i, 1, n + 1) printf("%d%c", sa.sa[i] + 1, i == _i - 1 ? '\n' : ' ');
    FOR (i, 2, n + 1) printf("%d%c", sa.ht[i], i == _i - 1 ? '\n' : ' ');
}
\end{minted}

\hypertarget{ux540eux7f00ux5e73ux8861ux6811}{%
\subsection{后缀平衡树}\label{ux540eux7f00ux5e73ux8861ux6811}}

后缀之间的大小由字典序定义,后缀平衡树就是一个维护这些后缀顺序的平衡树,即字符串
\(T\) 的后缀平衡树是 \(T\)
所有后缀的有序集合。后缀平衡树上的一个节点相当于原字符串的一个后缀。

特别地,后缀平衡树的中序遍历即为后缀数组。

对长度为 \(n\) 的字符串 \(T\)
建立其后缀平衡树,考虑逆序将其后缀加入后缀平衡树。

记后缀平衡树维护的集合为 \(X\),当前添加的后缀为
\(S\),则添加下一个后缀就是向 \(X\) 中加入
\(\texttt{c}S\)(亦可理解为后缀平衡树维护的字符串为 \(S\),下一步往
\(S\) 前加入一个字符
\(\texttt{c}\))。这一操作其实就是向平衡树中插入节点。
这里使用期望树高为 \(O(\log n)\) 的平衡树,例如替罪羊树或 Treap 等。

\hypertarget{ux505aux6cd5-1}{%
\subsubsection{做法 1}\label{ux505aux6cd5-1}}

插入时,暴力比较两个后缀之间的大小关系,从而判断之后是往哪一个子树添加。这样子,单次插入至多比较
\(O(\log n)\) 次,单次比较的时间复杂度至多为 \(O(n)\),一共
\(O(n\log n)\)。

一共会插入 \(n\) 次,所以该做法的时间复杂度存在上界 \(O(n^2 \log n)\)。

\hypertarget{ux505aux6cd5-2}{%
\subsubsection{做法 2}\label{ux505aux6cd5-2}}

注意到 \(\texttt{c}S\) 与 \(S\) 的区别仅在于 \(\texttt{c}\),且 \(S\)
已经属于 \(X\) 了,可以利用这一点来优化插入操作。

假设当前要比较 \(\texttt{c}S\) 与 \(A\) 两个字符串的大小,且
\(A, S \in X\)。每次比较时,首先比较两串的首字符。若首字符不等,则两串的大小关系就已经确定了;若首字符相等,那么就只需要判断去除首字符后两字符串的大小关系。而两串去除首字符后都已经属于
\(X\) 了,这时候可以借助平衡树 \(O(\log n)\)
求排名的操作来完成后续的比较。这样,单次插入的操作至多 \(O(\log^2 n)\)。

一共会插入 \(n\) 次,所以该做法的时间复杂度存在上界 \(O(n \log^2 n)\)。

\hypertarget{ux505aux6cd5-3}{%
\subsubsection{做法 3}\label{ux505aux6cd5-3}}

根据做法 2,如果能够 \(O(1)\)
判断平衡树中两个节点之间的大小关系,那么就可以在 \(O(n \log n)\)
的时间内完成后缀平衡树的构造。

记 \(val_i\) 表示节点 \(i\)
的值。如果在建平衡树时,每个节点多维护一个标记 \(tag_i\),使得若
\(tag_i > tag_j \iff val_i > val_j\),那么就可以根据 \(tag_i\) 的大小
\(O(1)\) 判断平衡树中两个节点的大小。

不妨令平衡树中每个节点对应一个实数区间,令根节点对应
\((0, 1)\)。对于节点 \(i\),记其对应的实数区间为 \((l, r)\),则
\(tag_i = \frac{l + r}{2}\),其左子树对应实数区间
\((l, tag_i)\),其右子树对应实数区间 \((tag_i, r)\)。易证 \(tag_i\)
满足上述要求。

由于使用了期望树高为 \(O(\log n)\)
的平衡树,所以精度是有一定保证的。实际实现时也可以用一个较大的区间来做,例如让根对应
\((0, 10^{18})\)。

\begin{minted}[fontsize=\footnotesize,breaklines,linenos]{cpp}
#include <bits/stdc++.h>
#define int long long
#define endl '\n'
using namespace std;

const int N = 1e6 + 10, INF = 1e18;
const double alpha = 0.75;

string t;

namespace SBT{
    #define ls tree[rt].lc
    #define rs tree[rt].rc

    struct node {
        int lc, rc;
        int sz;
    } tree[N];

    int cnt, root;
    double tag[N];

    void calc(int rt) {
        if(!rt) return;
        tree[rt].sz = tree[ls].sz + tree[rs].sz + 1;
    }

    bool can_rebuild(int rt) {
        return alpha * tree[rt].sz <= (double)max(tree[ls].sz, tree[rs].sz);
    }

    int ldr[N];

    void flatten(int &ldc, int rt) {
        if(!rt) return;
        flatten(ldc, ls);
        ldr[ldc++] = rt;
        flatten(ldc, rs);
    }

    int rebuild_bd(int l, int r, double lv, double rv) {
        if(l >= r) return 0;
        int mid = l + r >> 1;
        double mv = (lv + rv) / 2;
        tag[ldr[mid]] = mv;
        tree[ldr[mid]].lc = rebuild_bd(l, mid, lv, mv);
        tree[ldr[mid]].rc = rebuild_bd(mid + 1, r, mv, rv);
        calc(ldr[mid]);
        return ldr[mid];
    }

    void rebuild(int &rt, double lv, double rv) {
        int ldc = 0;
        flatten(ldc, rt);
        rt = rebuild_bd(0, ldc, lv, rv);
    }

    bool cmp(int x, int y) {
        if(t[x] != t[y]) return t[x] < t[y];
        return tag[x + 1] < tag[y + 1];
    }

    void insert(int &rt, int k, double lv, double rv) {
        if(!rt) {
            rt = k;
            tree[rt].sz = 1;
            tag[rt] = (lv + rv) / 2;
            ls = rs = 0;
            return;
        }
        if(cmp(k, rt)) insert(ls, k, lv, tag[rt]);
        else insert(rs, k, tag[rt], rv);
        calc(rt);
        if(can_rebuild(rt)) rebuild(rt, lv, rv);
    }

    void del(int &rt, int k, double lv, double rv) {
        if(!rt) return;
        if(rt == k) {
            if(!ls || !rs) {
                rt = ls | rs;
            } else {
                int nrt = ls, fa = rt;
                while(tree[nrt].rc) fa = nrt, tree[fa].sz--, nrt = tree[nrt].rc;
                if(fa == rt) tree[nrt].rc = rs;
                else tree[nrt].lc = ls, tree[nrt].rc = rs, tree[fa].rc = 0;
                rt = nrt; 
                tag[rt] = (lv + rv) / 2;
            }
        } else {
            double mv = (lv + rv) / 2;
            if(cmp(k, rt)) del(ls, k, lv, mv);
            else del(rs, k, mv, rv);
        }
        calc(rt);
        if(can_rebuild(rt)) rebuild(rt, lv, rv);
    }
} // Suffix Balanced Tree

int sa[N], tot = 0;

void get_sa(int rt) {
    using SBT::tree;
    if(!rt) return;
    get_sa(tree[rt].lc);
    sa[++tot] = rt;
    get_sa(tree[rt].rc);
}

inline void solve() {
    cin >> t; int n = t.size();
    t = '@' + t;
    for(int i = n; i >= 1; i--) SBT::insert(SBT::root, i, 0, INF); 
    get_sa(SBT::root);
    for(int i = 1; i <= n; i++) cout << sa[i] << " \n"[i == n];
}

signed main() {
    ios_base::sync_with_stdio(false), cin.tie(0);
    int t = 1; // cin >> t;
    while(t--) solve();
    return 0;
}
\end{minted}

\end{document}
