\PassOptionsToPackage{unicode=true}{hyperref} % options for packages loaded elsewhere
\PassOptionsToPackage{hyphens}{url}
%
\documentclass[]{article}
\usepackage{lmodern}
\usepackage{amssymb,amsmath}

% settings
\usepackage{minted}
%



\usepackage{ifxetex,ifluatex}
\usepackage{fixltx2e} % provides \textsubscript
\ifnum 0\ifxetex 1\fi\ifluatex 1\fi=0 % if pdftex
  \usepackage[T1]{fontenc}
  \usepackage[utf8]{inputenc}
  \usepackage{textcomp} % provides euro and other symbols
\else % if luatex or xelatex
  \usepackage{unicode-math}
  \defaultfontfeatures{Ligatures=TeX,Scale=MatchLowercase}
    \setmainfont[]{Source Han Serif CN}
    \setsansfont[]{Source Han Sans CN}
    \setmonofont[Mapping=tex-ansi]{Source Code Pro}
  \ifxetex
    \usepackage{xeCJK}
    \setCJKmainfont[]{Source Han Serif CN}
  \fi
  \ifluatex
    \usepackage[]{luatexja-fontspec}
    \setmainjfont[]{Source Han Serif CN}
  \fi
\fi
% use upquote if available, for straight quotes in verbatim environments
\IfFileExists{upquote.sty}{\usepackage{upquote}}{}
% use microtype if available
\IfFileExists{microtype.sty}{%
\usepackage[]{microtype}
\UseMicrotypeSet[protrusion]{basicmath} % disable protrusion for tt fonts
}{}
\IfFileExists{parskip.sty}{%
\usepackage{parskip}
}{% else
\setlength{\parindent}{0pt}
\setlength{\parskip}{6pt plus 2pt minus 1pt}
}
\usepackage{hyperref}
\hypersetup{
            pdfborder={0 0 0},
            breaklinks=true}
\urlstyle{same}  % don't use monospace font for urls
\usepackage[margin=2cm]{geometry}
\setlength{\emergencystretch}{3em}  % prevent overfull lines
\providecommand{\tightlist}{%
  \setlength{\itemsep}{0pt}\setlength{\parskip}{0pt}}
\setcounter{secnumdepth}{0}
% Redefines (sub)paragraphs to behave more like sections
\ifx\paragraph\undefined\else
\let\oldparagraph\paragraph
\renewcommand{\paragraph}[1]{\oldparagraph{#1}\mbox{}}
\fi
\ifx\subparagraph\undefined\else
\let\oldsubparagraph\subparagraph
\renewcommand{\subparagraph}[1]{\oldsubparagraph{#1}\mbox{}}
\fi


% set default figure placement to htbp
\makeatletter
\def\fps@figure{htbp}
\makeatother

\usepackage{minted}



\date{}

\title{\vspace{50mm} \huge Standard Code Library \\[20pt]}
\author{HeartFireY, eroengine, yezzz \\[10pt] Jiangxi Normal University}
\date{\today}


\begin{document}

\begin{titlepage}

\maketitle

\end{titlepage}

\newpage

\renewcommand\labelitemi{$\bullet$}

{
\setcounter{tocdepth}{3}
\tableofcontents
\newpage
}








\hypertarget{ux6570ux636eux7ed3ux6784}{%
\section{数据结构}\label{ux6570ux636eux7ed3ux6784}}

{[}TOC{]}

\hypertarget{ux79bbux6563ux5316}{%
\subsection{1.离散化}\label{ux79bbux6563ux5316}}

\hypertarget{ux6570ux7ec4ux7248}{%
\subsubsection{数组版}\label{ux6570ux7ec4ux7248}}

\begin{minted}[fontsize=\footnotesize,breaklines,linenos]{cpp}
for(int i = 1; i <= n; i++) std::cin >> a[i], b[i] = a[i];
std::sort(a + 1, a + 1 + n);
len = std::unique(a + 1, a + 1 + n) - a - 1;
auto query_pos = [&](int x) { return std::lower_bound(a + 1, a + 1 + len, x) - a; };
\end{minted}

\hypertarget{ux5411ux91cfux7248}{%
\subsubsection{向量版}\label{ux5411ux91cfux7248}}

\begin{minted}[fontsize=\footnotesize,breaklines,linenos]{cpp}
vector<int> a;
std::sort(a.begin(),a.end());
a.erase(unique(a.begin(),a.end()),a.end());
auto query_pos = [&](int x) { return lower_bound(a.begin(), a.end(), c) - a.begin() + 1;  
//查找下标从1开始
}
\end{minted}

\hypertarget{ux5e76ux67e5ux96c6}{%
\subsection{2.并查集}\label{ux5e76ux67e5ux96c6}}

\hypertarget{ux8defux5f84ux538bux7f29ux6309ux79e9ux5408ux5e76}{%
\subsubsection{路径压缩+按秩合并}\label{ux8defux5f84ux538bux7f29ux6309ux79e9ux5408ux5e76}}

\begin{minted}[fontsize=\footnotesize,breaklines,linenos]{cpp}
struct UnionFind {
    std::vector<int> par, rank, size;
    int c;
    UnionFind(int n) : par(n), rank(n, 0), size(n, 1), c(n) {
        for(int i = 0; i < n; ++i) par[i] = i;
    }
    int find(int i) { return (par[i] == i ? i : (par[i] = find(par[i]))); }
    bool same(int i, int j) { return find(i) == find(j); }
    int get_size(int i) { return size[find(i)]; }
    int count() { return c; }
    int merge(int i, int j) {
        if((i = find(i)) == (j = find(j))) return -1;
        else --c;
        if(rank[i] > rank[j]) std::swap(i, j);
        par[i] = j, size[j] += size[i];
        if(rank[i] == rank[j]) rank[j]++;
        return j;
    }
};
\end{minted}

\hypertarget{ux53efux56deux6edaux5e76ux67e5ux96c6}{%
\subsubsection{可回滚并查集}\label{ux53efux56deux6edaux5e76ux67e5ux96c6}}

\begin{itemize}
\tightlist
\item
  注意这个不是可持久化并查集
\item
  查找时不进行路径压缩
\item
  复杂度靠按秩合并解决
\end{itemize}

\begin{minted}[fontsize=\footnotesize,breaklines,linenos]{cpp}
namespace uf {
    int fa[maxn], sz[maxn];
    int undo[maxn], top;
    void init() { memset(fa, -1, sizeof fa); memset(sz, 0, sizeof sz); top = 0; }
    int findset(int x) { while (fa[x] != -1) x = fa[x]; return x; }
    bool join(int x, int y) {
        x = findset(x); y = findset(y);
        if (x == y) return false;
        if (sz[x] > sz[y]) swap(x, y);
        undo[top++] = x;
        fa[x] = y;
        sz[y] += sz[x] + 1;
        return true;
    }
    inline int checkpoint() { return top; }
    void rewind(int t) {
        while (top > t) {
            int x = undo[--top];
            sz[fa[x]] -= sz[x] + 1;
            fa[x] = -1;
        }
    }
}
\end{minted}

\hypertarget{stux8868}{%
\subsection{3.ST表}\label{stux8868}}

\begin{itemize}
\tightlist
\item
  预处理:\(O(n \log n)\)
\item
  查询:\(O(1)\)
\end{itemize}

\begin{minted}[fontsize=\footnotesize,breaklines,linenos]{cpp}
#include <bits/stdc++.h>
using namespace std;

int n, q;

namespace ST{
    const int N = 2000010;
    int stmax[N][22], stmin[N][22], mn[N], a[N];

    void init(int n){
        mn[0] = -1;
        for (int i = 1; i <= n; i++){
            mn[i] = ((i & (i - 1)) == 0) ? mn[i - 1] + 1 : mn[i - 1];
            stmax[i][0] = stmin[i][0] = a[i];
        }
        for (int j = 1; j <= mn[n]; j++)
            for (int i = 1; i + (1 << j) - 1 <= n; i++){
                stmax[i][j] = max(stmax[i][j - 1], stmax[i + (1 << (j - 1))][j - 1]);
                stmin[i][j] = min(stmin[i][j - 1], stmin[i + (1 << (j - 1))][j - 1]);
            }
    }

    inline int rmq_max(int L, int R){
        int k = mn[R - L + 1];
        return max(stmax[L][k], stmax[R - (1 << k) + 1][k]);
    }

    inline int rmq_min(int L, int R){
        int k = mn[R - L + 1];
        return min(stmin[L][k], stmin[R - (1 << k) + 1][k]);
    }
}

signed main(){
    cin >> n >> q;
    for(int i = 1; i <= n; i++) cin >> ST::a[i];
    ST::init(n);
    while(q--){
        int l, r; cin >> l >> r;
        cout << ST::rmq_max(l, r) << ' ' << ST::rmq_min(l, r) << endl;
    }
    return 0;
}
\end{minted}

\hypertarget{ux6811ux72b6ux6570ux7ec4fenwick}{%
\subsection{4.树状数组(Fenwick)}\label{ux6811ux72b6ux6570ux7ec4fenwick}}

\hypertarget{ux6734ux7d20ux6811ux72b6ux6570ux7ec4}{%
\subsubsection{朴素树状数组}\label{ux6734ux7d20ux6811ux72b6ux6570ux7ec4}}

\textbf{查询区间第K大需要权值树状数组!}

\begin{minted}[fontsize=\footnotesize,breaklines,linenos]{cpp}
namespace Fenwick{
    int tree[N], len;
    #define lowbit(x) ((x) & (-x))
    inline void init(){
        memset(tree, 0, sizeof(tree));
        for(int i = 1, tmp = 0; i <= len; i++){
            tree[i] += a[i];
            tmp = i + lowbit(i);
            if(tmp <= len) tree[tmp] += tree[i];
        }
    }

    inline void update(int i, int x){
        for(int pos = i; pos <= len; pos += lowbit(pos)) tree[pos] += x;
    }

    inline int getsum(int i, int ans = 0){
        for(int pos = i; pos; pos -= lowbit(pos)) ans += tree[pos];
        return ans;
    }

    inline int query(int l, int r){ return getsum(r) - getsum(l - 1); }

    //*查询区间第K大需要权值树状数组!
    int kth(int i){
        int cnt = 0, ret = 0;
        for(int i = log2(len); ~i; --i){
            ret += 1 << i;
            (ret >= len || cnt + tree[ret] >= i) ? (ret -= 1 << i) : (cnt += tree[ret]);
        }
        return ret + 1;
    }
}
\end{minted}

\hypertarget{ux533aux95f4ux52a0ux533aux95f4ux6c42ux548c}{%
\subsubsection{区间加/区间求和}\label{ux533aux95f4ux52a0ux533aux95f4ux6c42ux548c}}

若维护序列 \(a\) 的差分数组 \(b\),此时我们对 \(a\) 的一个前缀 \(r\)
求和,即 \(\sum_{i=1}^{r} a_i\),由差分数组定义得
\(a_i=\sum_{j=1}^i b_j\)

进行推导

\(\begin{aligned} &\sum_{i=1}^{r} a_i\\=&\sum_{i=1}^r\sum_{j=1}^i b_j\\=&\sum_{i=1}^r b_i\times(r-i+1) \\=&\sum_{i=1}^r b_i\times (r+1)-\sum_{i=1}^r b_i\times i \end{aligned}\)

区间和可以用两个前缀和相减得到,因此只需要用两个树状数组分别维护
\(\sum b_i\) 和 \(\sum i \times b_i\),就能实现区间求和。

\begin{minted}[fontsize=\footnotesize,breaklines,linenos]{cpp}
namespace Fenwick_Plus{
    #define lowbit(x) ((x) & (-x))
    #define MAXN $Array_Max_Sizes$
    int tree1[MAXN], tree2[MAXN], a[MAXN], len;
    //对两个树状数组进行更新
    inline void add(int i, int x){
        int x1 = i * x;
        for(int pos = i; pos <= len; pos += lowbit(pos)) tree1[pos] += x, tree2[pos] += x1;
    }
    //将区间加差分为两个前缀和
    inline void update(int l, int r, int x){
        add(l, x), add(r + 1, -x);
    }
    //对指定的树状数组求前n项和
    inline int getsum(int *tree, int i){
        int sum = 0;
        for(int pos = i; pos; pos -= lowbit(pos)) sum += tree[i];
        return sum;
    }
    //区间和查询
    inline int query(int l, int r){
        return (r + 1) * getsum(tree1, r) - l * getsum(tree1, l - 1) - (getsum(tree2, r) - getsum(tree2, l - 1));
    }
}
\end{minted}

\hypertarget{ux6734ux7d20ux7ebfux6bb5ux6811}{%
\subsection{4.朴素线段树}\label{ux6734ux7d20ux7ebfux6bb5ux6811}}

区间和:区间修改/单点修改/单点查询/区间查询

\begin{minted}[fontsize=\footnotesize,breaklines,linenos]{cpp}
#define SEGRG 1, 1, n //!此处定义范围,注意n取值

const int N = 1e5 + 10;

namespace SegmentTree{
    #define ls rt << 1
    #define rs rt << 1 | 1
    #define lson rt << 1, l, mid
    #define rson rt << 1 | 1, mid + 1, r
    int len, q, tree[N << 2], lazy[N << 2];
    inline void push_up(int rt) { tree[rt] = tree[ls] + tree[rs]; }
    
    inline void push_down(int rt, int m){
        if(!lazy[rt]) return;
        lazy[ls] += lazy[rt], lazy[rs] += lazy[rt];
        tree[ls] += lazy[rt] * (m - (m >> 1));
        tree[rs] += lazy[rt] * (m >> 1);
        lazy[rt] = 0;
    }

    static void build(int rt, int l, int r){
        tree[rt] = lazy[rt] = 0;
        if(l == r){
            tree[rt] = 0; //!build leaf_node here
            return; 
        }
        int mid = l + r >> 1;
        build(lson), build(rson);
        push_up(rt);
    }

    static void update_part(int rt, int l, int r, int L, int R, int val){
        if(l >= L && r <= R){
            lazy[rt] += val, tree[rt] += (r - l + 1) * val;
            return;
        }
        int mid = l + r >> 1;
        push_down(rt, r - l + 1);
        if(mid >= L) update_part(lson, L, R, val);
        if(mid < R) update_part(rson, L, R, val);
        push_up(rt);
    }

    static void update_point(int rt, int l, int r, int pos, int val){
        if(l == r){
            tree[rt] += val;
            return;
        }
        push_down(rt, r - l + 1);
        int mid = l + r >> 1;
        if(mid >= pos) update_point(lson, pos, val);
        else update_point(rson, pos, val);
        push_up(rt);
    }

    static int query(int rt, int l, int r, int L, int R){
        if(l >= L && r <= R) return tree[rt];
        int mid = l + r >> 1, ans = 0;
        push_down(rt, r - l + 1);
        if(mid >= L) ans += query(lson, L, R);
        if(mid < R) ans += query(rson, L, R);
        return ans;
    }
}
\end{minted}

\hypertarget{ux52a8ux6001ux5f00ux70b9ux7ebfux6bb5ux6811}{%
\subsection{5.动态开点线段树}\label{ux52a8ux6001ux5f00ux70b9ux7ebfux6bb5ux6811}}

\begin{minted}[fontsize=\footnotesize,breaklines,linenos]{cpp}
namespace SegmentTree{
    const int N = 3e5 + 10;
    int tree[N << 2], lson[N << 2], rson[N << 2], lazy[N << 2], tot = 0, root = 0;

    inline void push_up(int rt){ tree[rt] = tree[lson[rt]] + tree[rson[rt]]; }
    inline void push_down(int rt, int m){
        if(!lazy[rt]) return;
        if(!lson[rt]) lson[rt] = ++tot;
        if(!rson[rt]) rson[rt] = ++tot;
        lazy[lson[rt]] += lazy[rt], lazy[rson[rt]] += lazy[rt];
        tree[lson[rt]] += lazy[rt] * (m - (m >> 1));
        tree[rson[rt]] += lazy[rt] * (m >> 1);
        lazy[rt] = 0;
    }

    static void update_part(int &rt ,int l, int r, int L, int R, int val){
        if(!rt) rt = ++tot;
        if(l >= L && r <= R){
            lazy[rt] += val;
            tree[rt] += val * (r - l + 1);
            return;
        }
        push_down(rt, r - l + 1);
        int mid = l + r >> 1;
        if(mid >= L) update_part(lson[rt], l, mid, L, R, val);
        if(mid < R) update_part(rson[rt], mid + 1, r, L, R, val);
        push_up(rt);
    }

    static void update_point(int &rt, int l, int r, int pos, int val){
        if(!rt) rt = ++tot;
        if(l == r){
            tree[rt] += val;
            return;
        }
        int mid = l + r >> 1;
        if(mid >= pos) update_point(lson[rt], l, mid, pos, val);
        else update_point(rson[rt], mid + 1, r, pos, val);
        push_up(rt);
    }

    static int query(int rt, int l, int r, int L, int R){
        if(!rt) return 0;
        if(l >= L && r <= R) return tree[rt];
        push_down(rt, r - l + 1);
        int mid = l + r >> 1, ans = 0;
        if(mid >= L) ans += query(lson[rt], l, mid, L, R);
        if(mid < R) ans += query(rson[rt], mid + 1, r, L, R);
        return ans;
    }
}//DynamicSegmentTree
\end{minted}

\hypertarget{ux53efux6301ux4e45ux5316ux6743ux503cux7ebfux6bb5ux6811ux4e3bux5e2dux6811}{%
\subsection{6.可持久化权值线段树(主席树)}\label{ux53efux6301ux4e45ux5316ux6743ux503cux7ebfux6bb5ux6811ux4e3bux5e2dux6811}}

\hypertarget{ux6a21ux677fux9898ux6a21ux677fux5c01ux88c5}{%
\subsubsection{0.模板题/模板封装}\label{ux6a21ux677fux9898ux6a21ux677fux5c01ux88c5}}

给定排列\(p_1, p_2, p_3,\dots, p_n\),定义\(A_i\)表示在\(p_i\)左侧并比\(p_i\)小的数字个数,\(B_i\)表示在\(p_i\)右侧并比\(p_i\)小的数字个数,\(C_i = \min(A_i, B_i)\)。现在给定多个操作\((l, r)\),求每个操作,交换\((p_i, p_j)\)后的\(\sum C_i\)。

首先考虑如何处理初始时的\(C_i\)值,观察到以下性质:

\begin{itemize}
\tightlist
\item
  对于\(A_i\)值的求解过程类似求逆序对的思想,可以直接上树状数组维护,\(O(n\log n)\)求得全部的\(A_i\)
\item
  由于是排列,\(B_i = p_i - 1 - A_i\)可以\(O(1)\)求得
\item
  那么\(C_i = \min(A_i, B_i)\)也是\(O(1)\)得到的
\end{itemize}

由于每个询问相互独立,那么考虑交换\((p_l, p_r)\)操作对\(C_i\)的影响:

\begin{itemize}
\item
  对于\([1, l), (r , n]\)范围的数字,\(C_i\)值一定不影响。因为交换操作均在单侧进行
\item
  对于\(p_l\),交换到\(r\)位置后,\(A_l \rightarrow A_l + 区间[l,r]小于p_l的数字个数\),\(B_l’\)仍然可以直接求

  对于\(p_r\),交换到\(l\)位置后,\(A_r \rightarrow A_r - 区间[l ,r]小于p_r的数字个数\),\(B_r’\)仍然可以直接求

  如果我们在线询问(主席树维护),那么对于\(p_l,p_r\),实际上可以直接两个\(O(logn)\)重新求。
\item
  那么重点是对于\([l + 1, r - 1]\)区间内的数字的\(C_i\)值变化,如何维护?
\item
  对于\(p_l \leq p_i \leq p_r\),如果\(A_i \leq B_i\),则交换后\(A_i - 1, B_i + 1\),从而\(C_i - 1\)

  对于\(p_l \geq p_i \geq p_r\),如果\(A_i \geq B_i\),则交换后\(A_i + 1, B_i - 1\),从而\(C_i -1\)
\item
  对于\(p_l \leq p_i \leq p_r\),如果\(A_i - 1 \geq B_i + 1, A_i \geq B_i\),则交换后\(C_i + 1\)

  对于\(p_l \geq p_i \geq p_r\),如果\(A_i - 1 \leq B_i + 1, A_i \leq B_i\),则交换后\(C_i + 1\)
\end{itemize}

那么对于以上四种情况,我们可以分别用四棵主席树进行维护。同时,对于\(p_l, p_r\)的贡献计算还需要支持区间\(<K\)的数字个数查询,因此共需五棵主席树进行维护,复杂度\(O(m \times 4 \log n)\)。

\begin{minted}[fontsize=\footnotesize,breaklines,linenos]{cpp}
namespace PresidentTree{
    int root[N], sum[N << 5][5], lc[N << 5], rc[N << 5], cnt;
    #define ls l, mid
    #define rs mid + 1, r

    void update(int &rt, int pre, int l, int r, int x, bset5 inc){
        rt = ++cnt, lc[rt] = lc[pre], rc[rt] = rc[pre];
        for(int i = 0; i <= 5; i++) sum[rt][i] = sum[pre][i] + (inc[i] ? 1 : 0);
        if(l == r) return;
        int mid = l + r >> 1;
        if(x <= mid) update(lc[rt], lc[rt], l, mid, x, inc);
        else update(rc[rt], rc[rt], mid + 1, r, x, inc);
    }

    int query(int st, int ed, int l, int r, int L, int R, int id){
        if(l == L && r == R) return sum[ed][id] - sum[st][id];
        int mid = l + r >> 1;
        if(mid >= R) return query(lc[st], lc[ed], l, mid, L, R, id);
        else if(mid >= L) return query(lc[st], lc[ed], l, mid, L, mid, id) + query(rc[st], rc[ed], mid + 1, r, mid + 1, R, id);
        else return query(rc[st], rc[ed], mid + 1, r, L, R, id);
    }

}
\end{minted}

\hypertarget{ux4e3bux5e2dux6811ux7ef4ux62a4ux9759ux6001ux533aux95f4ux7b2ckux5927}{%
\subsubsection{\texorpdfstring{1.主席树维护静态区间第\(K\)大}{1.主席树维护静态区间第K大}}\label{ux4e3bux5e2dux6811ux7ef4ux62a4ux9759ux6001ux533aux95f4ux7b2ckux5927}}

\begin{minted}[fontsize=\footnotesize,breaklines,linenos]{cpp}
#include <bits/stdc++.h>
using namespace std;

const int N = 2e5 + 10;
int root[N], tot;
int lc[N << 5], rc[N << 5], sum[N << 5];
int a[N], b[N], n, m;

namespace cmt{
    void update(int &rt, int pre, int l, int r, int pos, int v){
        rt = ++tot, lc[rt] = lc[pre], rc[rt] = rc[pre], sum[rt] = sum[pre] + 1;
        if(l == r) return;
        int mid = l + r >> 1;
        if(pos <= mid) update(lc[rt], lc[pre], l, mid, pos, v);
        else update(rc[rt], rc[pre], mid + 1, r, pos, v);
    }

    int query(int ql, int qr, int l, int r, int k){
        if(l == r) return l;
        int mid = l + r >> 1, summ = sum[lc[qr]] - sum[lc[ql]];
        if(summ >= k) return query(lc[ql], rc[qr], l, mid, k);
        else return query(rc[ql], rc[qr], mid + 1, r, k - summ);
    }
}

signed main(){
    cin >> n >> m;
    for(int i = 1; i <= n; i++){
        cin >> a[i]; 
        b[i] = a[i];
    }
    sort(b + 1, b + 1 + n);
    int n_1 = unique(b + 1, b + 1 + n) - (b + 1);
    for(int i = 1; i <= n; i++) cmt::update(root[i], root[i - 1], 1, n_1, lower_bound(b + 1, b + 1 + n_1, a[i]) - b, 1);
    for(int i = 1; i <= m; i++){
        int l, r, k; cin >> l >> r >> k;
        cout << b[cmt::query(root[l - 1], root[r], 1, n_1, k)] << endl;
    }
    return 0;
}
\end{minted}

\hypertarget{ux4e3bux5e2dux6811ux7ef4ux62a4ux7ebfux6bb5ux6811ux533aux95f4ux4feeux6539ux6807ux8bb0ux6c38ux4e45ux5316}{%
\subsubsection{2.主席树维护线段树区间修改(标记永久化)}\label{ux4e3bux5e2dux6811ux7ef4ux62a4ux7ebfux6bb5ux6811ux533aux95f4ux4feeux6539ux6807ux8bb0ux6c38ux4e45ux5316}}

主席树可以维护线段树的区间修改,但是要求线段树动态开点。

在区间更新的时候,对每个点打永久化标记,对于跨越区间的点需要直接修改。查询时找到对应的完全覆盖区间加上这个区间的标记*(区间长度)

\begin{minted}[fontsize=\footnotesize,breaklines,linenos]{cpp}
#include <bits/stdc++.h>
#define ll long long
using namespace std;

const int N = 1e5 + 10;
int root[N], lc[N << 6], rc[N << 6], tot = 0, cur = 0;
ll sum[N << 6], lazy[N << 6];
int n, m;

void build(int &rt, int l, int r){
    rt = ++tot, lazy[rt] = 0;
    if(l == r){
        scanf("%lld", &sum[rt]);
        return;
    }
    int mid = l + r >> 1;
    build(lc[rt], l, mid);
    build(rc[rt], mid + 1, r);
    sum[rt] = sum[lc[rt]] + sum[rc[rt]];
}

void update(int &rt, int pre, int l, int r, int L, int R, int c){
    rt = ++tot, lc[rt] = lc[pre], rc[rt] = rc[pre], lazy[rt] = lazy[pre], sum[rt] = sum[pre] + 1ll * (min(r, R) - max(l, L) + 1) * c;
    if(L >= l && R <= r){
        lazy[rt] += c;
        return;
    }
    int mid = L + R >> 1;
    if(l <= mid) update(lc[rt], lc[pre], l, r, L, mid, c);
    if(r > mid) update(rc[rt], rc[pre], l, r, mid + 1, R, c);
}

ll query(int rt, int L, int R, int l, int r){
    if(L >= l && R <= r) return sum[rt];
    int mid = L + R >> 1;
    ll ans = lazy[rt] * (min(r, R) - max(l, L) + 1);
    if(l <= mid) ans += query(lc[rt], L, mid, l, r);
    if(r > mid) ans += query(rc[rt], mid + 1, R, l, r);
    return ans;
}

signed main(){
    while(scanf("%d%d", &n, &m) != EOF){
        char op[10]; 
        cur = tot = 0;
        build(root[0], 1, n);
        while(m--){
            scanf("%s", op);           
            if(op[0] == 'C'){
                int l, r, c;
                scanf("%d%d%d", &l, &r, &c);
                cur++;
                update(root[cur], root[cur - 1], l, r, 1, n, c);
            }
            else if(op[0] == 'Q'){
                int l, r;
                scanf("%d%d", &l, &r);
                printf("%lld\n", query(root[cur], 1, n, l, r));
            }
            else if(op[0] == 'H'){
                int l, r, h;
                scanf("%d%d%d", &l, &r, &h);
                printf("%lld\n", query(root[h], 1, n, l, r));
            }
            else if(op[0] == 'B'){
                scanf("%d", &cur);
            }
        }
    }
    return 0; 
}
\end{minted}

\hypertarget{ux6c42ux533aux95f4ux5185ux4e0dux540cux7684ux6570ux5b57ux4e2aux6570ux6c42ux533aux95f4ux5927ux4e8ekux7684ux6570ux5b57ux6709ux591aux5c11}{%
\subsubsection{\texorpdfstring{3.求区间内不同的数字个数/求区间大于\(K\)的数字有多少}{3.求区间内不同的数字个数/求区间大于K的数字有多少}}\label{ux6c42ux533aux95f4ux5185ux4e0dux540cux7684ux6570ux5b57ux4e2aux6570ux6c42ux533aux95f4ux5927ux4e8ekux7684ux6570ux5b57ux6709ux591aux5c11}}

求区间内不同数字个数的问题可以转化为求区间内小于等于\(K\)的数字个数:

对于每个数字记录下一个最近的相同数字下标\(nxt[i]\),那么查询区间\([L,R]\)内不同数字的个数实际上就是在查询区间内\(nxt[i] > R\)的个数(下一个相同的数字位于区间之外)。那么现在不难发现对于给定区间\([l,r]\),如果\(nxt[i] > r,\ i \in [l,r]\),那么表示与\(i\)相同数字点位于区间之外。那么求不同数字个数问题便转化为给定求所有满足\(l<=i<=r,\ nxt[i]>r\)的个数。

那么我们只需要处理出\(nxt\)数组,然后用主席树对每个节点维护权值数组,然后区间查询数目即可。

\begin{minted}[fontsize=\footnotesize,breaklines,linenos]{cpp}
#include <bits/stdc++.h>
using namespace std;

const int N = 1e6 + 10;
int a[N], head[N], nxt[N];
int root[N], sum[N << 5], lc[N << 5], rc[N << 5], cnt;

inline int read(){
    int f = 1, x = 0; char s = getchar(); 
    while(s < '0'||s > '9'){ if(s =='-') f = -1; s = getchar(); } 
    while(s >= '0' && s <= '9'){ x = x * 10 + s - '0'; s = getchar();}
    return x *= f; 
}

void build(int &rt, int l, int r){
    rt = ++cnt;
    if(l == r) return;
    int mid = l + r >> 1;
    build(lc[rt], l, mid);
    build(rc[rt], mid + 1, r);
}

void update(int &rt, int pre, int l, int r, int x){
    rt = ++cnt, lc[rt] = lc[pre], rc[rt] = rc[pre], sum[rt] = sum[pre] + 1;
    if(l == r) return;
    int mid = l + r >> 1;
    if(x <= mid) update(lc[rt], lc[pre], l, mid, x);
    else update(rc[rt], rc[pre], mid + 1, r, x);
}

int query(int L, int R, int l, int r, int k){
    if(l == r) return sum[R] - sum[L];
    int mid = l + r >> 1, ans = 0;
    if(k <= mid) ans += query(lc[L], lc[R], l, mid, k) + sum[rc[R]] - sum[rc[L]];
    else ans += query(rc[L], rc[R], mid + 1, r, k);
    return ans;
}

signed main(){
    int n = 0;
    n = read();
    for(int i = 1; i <= n; i++){
        a[i] = read();  //cin >> a[i];
        if(head[a[i]]) nxt[head[a[i]]] = i;
        head[a[i]] = i;
    }
    for(int i = 1; i <= n; i++)
        if(!nxt[i]) nxt[i] = n + 1;
    
    build(root[0], 1, n + 1);
    for(int i = 1; i <= n; i++) update(root[i], root[i - 1], 1, n + 1, nxt[i]);

    int m = read();
    while(m--){
        int l = read(), r = read();
        printf("%d\n", query(root[l - 1], root[r], 1, n + 1, r + 1));
    }
    return 0;
}
\end{minted}

\hypertarget{ux6c42ux533aux95f4ux5c0fux4e8eux7b49ux4e8ekux7684ux6570ux5b57ux4e2aux6570ux4e8cux5206ux67e5ux8be2}{%
\subsubsection{\texorpdfstring{4.求区间小于等于\(K\)的数字个数(二分查询)}{4.求区间小于等于K的数字个数(二分查询)}}\label{ux6c42ux533aux95f4ux5c0fux4e8eux7b49ux4e8ekux7684ux6570ux5b57ux4e2aux6570ux4e8cux5206ux67e5ux8be2}}

建立权值数组,对每个节点维护一颗主席树。查询为单点查询,查询数字对应的权值数组的个数。然后对于每个询问,我们在区间内二分枚举所有可能的数字\(k\),然后查询区间第\(k\)小。反复查询求得一个满足\(k \leq h\)的最大\(k\)。那么这个\(k\)就是我们想要得到的答案。

\begin{minted}[fontsize=\footnotesize,breaklines,linenos]{cpp}
#include <bits/stdc++.h>
#define ll long long
using namespace std;

const int N = 1e5 + 10;

ll root[N], sum[N << 5], lc[N << 5], rc[N << 5], tot = 0;
ll a[N], b[N];

inline void update(ll &rt, ll pre, ll l, ll r, ll k){
    rt = ++tot, lc[rt] = lc[pre], rc[rt] = rc[pre], sum[rt] = sum[pre] + 1;
    ll mid = l + r >> 1;
    if(l == r) return;
    if(k <= mid) update(lc[rt], lc[pre], l, mid, k);
    else update(rc[rt], rc[pre], mid + 1, r, k);
}

ll query(ll u, ll v, ll L, ll R, ll k){
    if(L == R) return L;
    ll mid = L + R >> 1;
    ll res = sum[lc[v]] - sum[lc[u]];
    if(res >= k) return query(lc[u], lc[v], L, mid, k);
    else return query(rc[u], rc[v], mid + 1, R, k - res);
}

signed main(){
    ios_base::sync_with_stdio(false), cin.tie(0), cout.tie(0);
    int t = 0, T = 0; cin >> t;
    while(t--){
        ll n, m; cin >> n >> m;
        for(int i = 1; i <= n; i++){
            cin >> a[i]; b[i] = a[i];
        }
        sort(b + 1, b + 1 + n);
        ll sz = unique(b + 1, b + 1 + n) - b - 1;
        for(int i = 1; i <= n; i++){
            ll x = lower_bound(b + 1, b + 1 + sz, a[i]) - b;
            update(root[i], root[i - 1], 1, sz, x);
        }
        cout << "Case " << ++T << ":" << endl;
        while(m--){
            ll u, v, k; cin >> u >> v >> k;
            u++, v++;
            ll ans = 0, L = 0, R = v - u + 1;
            while(L < R){
                ll mid = (L + R + 1) >> 1;
                ll t = query(root[u - 1], root[v], 1, sz, mid);
                if(b[t] <= k) L = mid;
                else R = mid - 1;
            }
            cout << L << endl;
        }
    }
    return 0;
}
\end{minted}

\hypertarget{ux6c42ux533aux95f4mex}{%
\subsubsection{5.求区间Mex}\label{ux6c42ux533aux95f4mex}}

给定\(n\)长度的数组,\(\{a_1, a_2,...,a_n\}\),以及\(m\)次询问,每次给出一个数对\((l, r)\)表示区间起点终点,要求对于给定的询问,回答在该区间内最小未出现的数字。

建立权值数组,对于每个点建立一棵主席树,维护权值最后一次出现的位置,那么对于查询\([l, r]\)就是查找第\(r\)棵树上出现位置小于\(l\)的权值,那么只需要维护最后一次出现位置的最小值即可。

主席树解决该问题属于在线算法。这种题目可以用莫队强制离线处理。

\begin{minted}[fontsize=\footnotesize,breaklines,linenos]{cpp}
#include <bits/stdc++.h>
using namespace std;

const int N = 2e5 + 10;

int tot, root[N], tree[N << 5], lc[N << 5], rc[N << 5];

void update(int &rt, int pre, int l, int r, int x, int val){
    rt = ++tot, lc[rt] = lc[pre], rc[rt] = rc[pre];
    if(l == r){
        tree[rt] = val; 
        return;
    }
    int mid = l + r >> 1;
    if(x <= mid) update(lc[rt], lc[pre], l, mid, x, val);
    else update(rc[rt], rc[pre], mid + 1, r, x, val);
    tree[rt] = min(tree[lc[rt]], tree[rc[rt]]);
}

int query(int rt, int ql, int l, int r){
    if(l == r) return l;
    int mid = l + r >> 1;
    if(tree[lc[rt]] < ql) return query(lc[rt], ql, l, mid);
    else return query(rc[rt], ql, mid + 1, r);
}


signed main(){
    ios_base::sync_with_stdio(false), cin.tie(0), cout.tie(0);
    int n, m; cin >> n >> m;
    for(int i = 1, x; i <= n; i++){
        cin >> x; x++;
        if(x > n) root[i] = root[i - 1];
        else update(root[i], root[i - 1], 1, n + 1, x, i);
    }
    while(m--){
        int l, r; cin >> l >> r;
        cout << query(root[r], l, 1, n + 1) - 1 << endl;
    }
    return 0;
}
\end{minted}

\hypertarget{ux6c42ux533aux95f4ux5185ux51faux73b0ux6b21ux6570ux5927ux4e8ekux6b21ux7684ux6700ux524dux6570}{%
\subsubsection{6.求区间内出现次数大于\textgreater=k次的最前数}\label{ux6c42ux533aux95f4ux5185ux51faux73b0ux6b21ux6570ux5927ux4e8ekux6b21ux7684ux6700ux524dux6570}}

对于给定的序列,输出待查询区间内出现次数严格大于区间长度一半的数字。

\textbf{思路:}考虑对查询过程进行剪枝,排除非法子树,向合法子树搜索。

首先考虑非法状态:因为对于主席树上任意一个节点,其代表的意义是管辖区间内数字的个数。因此对于主席树上某个节点,如果其代表区间数字的数目比区间长度的一半(也就是\(\frac{r - l + 1}{2}\))要小,那么子区间不回再出现满足该条件的数,在这种情况下可以直接返回\(0\)。

剩下的部分就是查询的板子。非法状态实际上就是在对查询过程进行剪枝。

\begin{minted}[fontsize=\footnotesize,breaklines,linenos]{cpp}
#include <bits/stdc++.h>
using namespace std;

const int N = 5e5 + 10;

int a[N], b[N];
int tot, root[N << 5], sum[N << 5], lc[N << 5], rc[N << 5];

void update(int &rt, int pre, int l, int r, int v){
    rt = ++tot, lc[rt] = lc[pre], rc[rt] = rc[pre], sum[rt] = sum[pre] + 1;
    if(l == r) return;
    int mid = l + r >> 1;
    if(v <= mid) update(lc[rt], lc[pre], l, mid, v);
    else update(rc[rt], rc[pre], mid + 1, r, v);
}

int query(int L, int R, int l, int r, int k){
    if(sum[R] - sum[L] <= k) return 0;
    if(l == r) return l;
    int now = sum[lc[R]] - sum[lc[L]], mid = l + r >> 1;
    if(now > k) return query(lc[L], lc[R], l, mid, k);
    else return query(rc[L], rc[R], mid + 1, r, k);
}

signed main(){
    ios_base::sync_with_stdio(false), cin.tie(0), cout.tie(0);
    int n, q; cin >> n >> q;
    for(int i = 1; i <= n; i++){
        cin >> a[i];
        b[i] = a[i];
    }
    sort(b + 1, b + n + 1);
    int m = unique(b + 1, b + n + 1) - b - 1;
    for(int i = 1; i <= n; i++){
        int x = lower_bound(b + 1, b + m + 1, a[i]) - b;
        update(root[i], root[i - 1], 1, m, x);
    }
    while(q--){
        int l, r; cin >> l >> r;
        int k = (r - l + 1) >> 1;
        cout << b[query(root[l - 1], root[r], 1, m, k)] << endl;
    }
    return 0;
}
\end{minted}

\hypertarget{ux4e3bux5e2dux6811ux6811ux4e0aux8defux5f84}{%
\subsubsection{7.主席树+树上路径}\label{ux4e3bux5e2dux6811ux6811ux4e0aux8defux5f84}}

给定一棵 \(n\) 个节点的树,每个点有一个权值。有 \(m\) 个询问,每次给你
\(u,v,k\) 你需要回答 \(u \text{ xor last}\) 和 \(v\) 这两个节点间第
\(k\) 小的点权。

动态查询树上区间点权第\(k\)小,且查询之间具有关系,因此考虑建立主席树维护区间信息。

首先回顾主席树维护线性区间第\(k\)大/小时我们的处理思路:

对于全局区间第\(k\)小时,我们建立全局权值线段树,维护的区间和表示某点子树中点的个数。那么我们在寻找区间第\(k\)小时,只需要左右二分寻找即可。而对于某个区间的第\(k\)小,一个朴素的方式便是我们每次都建立一颗线段树,但显然这样是不明智的算法。那么我们是如何查询这个区间第\(k\)小的呢?

对于这个维护的过程我们很容易联想到前缀和的概念,我们可以先离线建树,对于每个点建立一棵主席树,维护\([1, i]\)区间,那么对于区间查询\([l, r]\)时,我们只需要查询到区间\([1, l - 1]\)和\([1, r]\)即可取得区间的信息,实现区间查询。

然后分析样例,作出样例所示的树(假设以\(1\)为根节点):

那么我们可以发现,对于树上区间查询,我们也可以利用类似于线性区间查询的思路进行解决,但是由于树的结构限制,我们把线性区间的前缀和改为树上前缀和的形式:
\(query(u,\ v)\ =\ sum[u]+sum[v]-sum[lca(u,\ v)]-sum[fa[lca(u,\ v)]]\)
下面我们来说明这个式子:

如上,从根节点到\(5\)号节点的路径+从根节点到\(7\)号节点的路径重复了两次,那么我们要减去重叠的信息:对于根节点到交点父节点的信息均重复两次,到交点的信息重复一次(因为交点在链上,需要保留一次信息),因此前缀和形式便是\(sum[u]+sum[v]-sum[lca(u,\ v)]-sum[fa[lca(u,\ v)]]\)。

\begin{minted}[fontsize=\footnotesize,breaklines,linenos]{cpp}
#include <bits/stdc++.h>
#define id(x) (lower_bound(b + 1, b + le + 1, a[x]) - b)
#define rid(x) (b[x])
using namespace std;

const int N = 1e5 + 10;

int n, m, le, ans = 0, lastans = 0;
int a[N], b[N], f[N][19], dep[N];
vector<int> g[N];

struct node{ int sum, lc, rc; }tree[N << 5];
int root[N], cnt = 0;


void build(node &rt, int l, int r){
    rt.sum = 0;
    if(l == r) return;
    int mid = l + r >> 1;
    build(tree[rt.lc = ++cnt], l, mid);
    build(tree[rt.rc = ++cnt], mid + 1, r);
}

inline void update(node pre, node &rt, int l, int r, int p){
    rt.sum = pre.sum + 1;
    if(l == r) return;
    int mid = l + r >> 1;
    if(p <= mid) update(tree[pre.lc], tree[rt.lc = ++cnt], l, mid, p), rt.rc= pre.rc;
    else update(tree[pre.rc], tree[rt.rc = ++cnt], mid + 1, r, p), rt.lc = pre.lc;
}

inline int query(node u, node v, node lca, node lca_fa, int l, int r, int k){
    if(l == r) return l;
    int sum = tree[u.lc].sum + tree[v.lc].sum - tree[lca.lc].sum - tree[lca_fa.lc].sum;
    int mid = l + r >> 1;
    if(sum >= k) return query(tree[u.lc], tree[v.lc], tree[lca.lc], tree[lca_fa.lc], l, mid, k);
    return query(tree[u.rc], tree[v.rc], tree[lca.rc], tree[lca_fa.rc], mid + 1, r, k - sum);
}

inline void dfs(int u, int fa){
    update(tree[root[fa]], tree[root[u] = ++cnt], 1, le, id(u));
    f[u][0] = fa;
    dep[u] = dep[fa] + 1;
    for(register int i = 1; i <= 18; i++) f[u][i] = f[f[u][i - 1]][i - 1];
    for(auto v : g[u]){
        if(v == fa) continue;
        dfs(v, u);
    }
}

inline int lca(int u, int v){
    if(dep[u] < dep[v]) swap(u, v);
    for(register int i = 18; i >= 0; i--)
        if(dep[f[u][i]] >= dep[v]) u = f[u][i];
    if(u == v) return u;
    for(register int i = 18; i >= 0; i--)
        if(f[u][i] != f[v][i]) u = f[u][i], v = f[v][i];
    return f[u][0];
}

inline int querypath(int u, int v, int k){
    int lcaa = lca(u, v);
    return rid(query(tree[root[u]], tree[root[v]], tree[root[lcaa]], tree[root[f[lcaa][0]]], 1, le, k));
}

signed main(){
    ios_base::sync_with_stdio(false), cin.tie(0), cout.tie(0);
    //freopen("stdin.in", "r", stdin);
    //freopen("stdout.out", "w", stdout);
    cin >> n >> m;
    for(register int i = 1; i <= n; i++){ cin >> a[i]; b[i] = a[i]; }
    for(register int i = 1, u, v; i < n; i++){
        cin >> u >> v;
        g[u].push_back(v), g[v].push_back(u);
    }
    sort(b + 1, b + 1 + n);
    le = unique(b + 1, b + n + 1) - (b + 1);
    build(tree[root[0] = ++cnt], 1, le);
    dfs(1, 0);
    while(m--){
        int u, v, k; cin >> u >> v >> k;
        ans = querypath(u ^ lastans, v, k);
        cout << ans << endl;
        lastans = ans;
    }
    return 0;
}
\end{minted}

\hypertarget{ux6811ux5957ux6811}{%
\subsection{7.树套树}\label{ux6811ux5957ux6811}}

\hypertarget{ux6811ux72b6ux6570ux7ec4ux5957ux4e3bux5e2dux6811}{%
\subsubsection{树状数组套主席树}\label{ux6811ux72b6ux6570ux7ec4ux5957ux4e3bux5e2dux6811}}

动态区间第\(k\)大

\begin{minted}[fontsize=\footnotesize,breaklines,linenos]{cpp}
typedef vector<int> VI;
struct TREE {
#define mid ((l + r) >> 1)
#define lson l, mid
#define rson mid + 1, r
    struct P {
        int w, ls, rs;
    } tr[maxn * 20 * 20];
    int sz = 1;
    TREE() { tr[0] = {0, 0, 0}; }
    int N(int w, int ls, int rs) {
        tr[sz] = {w, ls, rs};
        return sz++;
    }
    int add(int tt, int l, int r, int x, int d) {
        if (x < l || r < x) return tt;
        const P& t = tr[tt];
        if (l == r) return N(t.w + d, 0, 0);
        return N(t.w + d, add(t.ls, lson, x, d), add(t.rs, rson, x, d));
    }
    int ls_sum(const VI& rt) {
        int ret = 0;
        FOR (i, 0, rt.size())
            ret += tr[tr[rt[i]].ls].w;
        return ret;
    }
    inline void ls(VI& rt) { transform(rt.begin(), rt.end(), rt.begin(), [&](int x)->int{ return tr[x].ls; }); }
    inline void rs(VI& rt) { transform(rt.begin(), rt.end(), rt.begin(), [&](int x)->int{ return tr[x].rs; }); }
    int query(VI& p, VI& q, int l, int r, int k) {
        if (l == r) return l;
        int w = ls_sum(q) - ls_sum(p);
        if (k <= w) {
            ls(p); ls(q);
            return query(p, q, lson, k);
        }
        else {
            rs(p); rs(q);
            return query(p, q, rson, k - w);
        }
    }
} tree;
struct BIT {
    int root[maxn];
    void init() { memset(root, 0, sizeof root); }
    inline int lowbit(int x) { return x & -x; }
    void update(int p, int x, int d) {
        for (int i = p; i <= m; i += lowbit(i))
            root[i] = tree.add(root[i], 1, m, x, d);
    }
    int query(int l, int r, int k) {
        VI p, q;
        for (int i = l - 1; i > 0; i -= lowbit(i)) p.push_back(root[i]);
        for (int i = r; i > 0; i -= lowbit(i)) q.push_back(root[i]);
        return tree.query(p, q, 1, m, k);
    }
} bit;

void init() {
    m = 10000;
    tree.sz = 1;
    bit.init();
    FOR (i, 1, m + 1)
        bit.update(i, a[i], 1);
}
\end{minted}

\hypertarget{k-d-tree}{%
\subsection{8.K-D Tree}\label{k-d-tree}}

在一个初始值全为 \(0\) 的 \(n\times n\) 的二维矩阵上,进行 \(q\)
次操作,每次操作为以下两种之一:

\begin{enumerate}
\def\labelenumi{\arabic{enumi}.}
\tightlist
\item
  \texttt{1\ x\ y\ A}:将坐标 \((x,y)\) 上的数加上 \(A\)。
\item
  \texttt{2\ x1\ y1\ x2\ y2}:输出以 \((x_1,y_1)\)
  为左下角,\((x_2,y_2)\) 为右上角的矩形内(包括矩形边界)的数字和。
\end{enumerate}

强制在线。内存限制 \texttt{20M}。保证答案及所有过程量在 \texttt{int}
范围内。

构建 2-D Tree,支持两种操作:添加一个 \(2\)
维点;查询矩形区域内的所有点的权值和。可以使用 \textbf{带重构} 的 k-D
Tree 实现。

在查询矩形区域内的所有点的权值和时,仍然需要记录子树内每一维度上的坐标的最大值和最小值。如果当前子树对应的矩形与所求矩形没有交点,则不继续搜索其子树;如果当前子树对应的矩形完全包含在所求矩形内,返回当前子树内所有点的权值和;否则,判断当前点是否在所求矩形内,更新答案并递归在左右子树中查找答案。

已经证明,如果在 \(2-D\)
树上进行矩阵查询操作,已经被完全覆盖的子树不会继续查询,则单次查询时间复杂度是最优
\(O(\log n)\),最坏 \(O(\sqrt n)\) 的。将结论扩展到 \(k\)
维的情况,则最坏时间复杂度是 \(O(n^{1-\frac 1 k})\) 的。

\begin{minted}[fontsize=\footnotesize,breaklines,linenos]{cpp}
const int maxn = 200010;
int n, op, xl, xr, yl, yr, lstans;

struct node{ int x, y, v; } s[maxn];

bool cmp1(int a, int b) { return s[a].x < s[b].x; }

bool cmp2(int a, int b) { return s[a].y < s[b].y; }

double a = 0.725;
int rt, cur, d[maxn], lc[maxn], rc[maxn], L[maxn], R[maxn], D[maxn], U[maxn],
    siz[maxn], sum[maxn];
int g[maxn], t;

void print(int x){
    if (!x) return;
    print(lc[x]);
    g[++t] = x;
    print(rc[x]);
}

void maintain(int x){
    siz[x] = siz[lc[x]] + siz[rc[x]] + 1;
    sum[x] = sum[lc[x]] + sum[rc[x]] + s[x].v;
    L[x] = R[x] = s[x].x;
    D[x] = U[x] = s[x].y;
    if (lc[x])
        L[x] = min(L[x], L[lc[x]]), R[x] = max(R[x], R[lc[x]]),
        D[x] = min(D[x], D[lc[x]]), U[x] = max(U[x], U[lc[x]]);
    if (rc[x])
        L[x] = min(L[x], L[rc[x]]), R[x] = max(R[x], R[rc[x]]),
        D[x] = min(D[x], D[rc[x]]), U[x] = max(U[x], U[rc[x]]);
}

int build(int l, int r){
    if (l > r) return 0;
    int mid = (l + r) >> 1;
    double av1 = 0, av2 = 0, va1 = 0, va2 = 0;
    for (int i = l; i <= r; i++)
        av1 += s[g[i]].x, av2 += s[g[i]].y;
    av1 /= (r - l + 1);
    av2 /= (r - l + 1);
    for (int i = l; i <= r; i++)
        va1 += (av1 - s[g[i]].x) * (av1 - s[g[i]].x), va2 += (av2 - s[g[i]].y) * (av2 - s[g[i]].y);
    if (va1 > va2) nth_element(g + l, g + mid, g + r + 1, cmp1), d[g[mid]] = 1;
    else nth_element(g + l, g + mid, g + r + 1, cmp2), d[g[mid]] = 2;
    lc[g[mid]] = build(l, mid - 1);
    rc[g[mid]] = build(mid + 1, r);
    maintain(g[mid]);
    return g[mid];
}

void rebuild(int &x){
    t = 0;
    print(x);
    x = build(1, t);
}

bool bad(int x) { return a * siz[x] <= (double)max(siz[lc[x]], siz[rc[x]]); }

void insert(int &x, int v){
    if (!x){
        x = v;
        maintain(x);
        return;
    }
    if (d[x] == 1){
        if (s[v].x <= s[x].x) insert(lc[x], v);
        else insert(rc[x], v);
    }
    else{
        if (s[v].y <= s[x].y) insert(lc[x], v);
        else insert(rc[x], v);
    }
    maintain(x);
    if (bad(x)) rebuild(x);
}

int query(int x){
    if (!x || xr < L[x] || xl > R[x] || yr < D[x] || yl > U[x]) return 0;
    if (xl <= L[x] && R[x] <= xr && yl <= D[x] && U[x] <= yr) return sum[x];
    int ret = 0;
    if (xl <= s[x].x && s[x].x <= xr && yl <= s[x].y && s[x].y <= yr) ret += s[x].v;
    return query(lc[x]) + query(rc[x]) + ret;
}

int main(){
    scanf("%d", &n);
    while (~scanf("%d", &op)){
        if (op == 1){
            cur++, scanf("%d%d%d", &s[cur].x, &s[cur].y, &s[cur].v);
            s[cur].x ^= lstans;
            s[cur].y ^= lstans;
            s[cur].v ^= lstans;
            insert(rt, cur);
        }
        if (op == 2){
            scanf("%d%d%d%d", &xl, &yl, &xr, &yr);
            xl ^= lstans;
            yl ^= lstans;
            xr ^= lstans;
            yr ^= lstans;
            printf("%d\n", lstans = query(rt));
        }
        if (op == 3) return 0;
    }
}
/* Test Case
//INPUT
4
1 2 3 3
2 1 1 3 3
1 1 1 1
2 1 1 0 7
3
//OUTPUT
3
5
*/
\end{minted}

\hypertarget{trieux5b57ux5178ux6811}{%
\subsection{9.Trie(字典树)}\label{trieux5b57ux5178ux6811}}

\hypertarget{a.0-1-trie}{%
\subsubsection{A.0-1 Trie}\label{a.0-1-trie}}

01字典树主要用于解决求异或最值的问题

\begin{minted}[fontsize=\footnotesize,breaklines,linenos]{cpp}
namespace Trie01{
    int tot, nxt[20*N][2];
    inline void clear() { tot = nxt[0][0] = nxt[0][1] = 0; }
    void insert(int x) {
        int rt = 0;
        for (int i = 20; i >= 0; --i) {
            int cur = (x >> i & 1);
            if (!nxt[rt][cur]) {
                nxt[rt][cur] = ++tot;
                nxt[tot][0] = nxt[tot][1] = 0;    
            }
            rt = nxt[rt][cur];
        }
    }

    int query_min(int x) {
        int rt = 0, ans = 0;
        for (int i = 20; i >= 0; -- i) {
            int cur = (x >> i & 1), need = cur;
            if (!nxt[rt][need]) ans = (ans << 1) | 1, need = !need;
            else ans = (ans << 1);
            rt = nxt[rt][need];
        }
        return ans;
    }

    int query_max(int x) {
        int rt = 0, ans = 0;
        for (int i = 20; i >= 0; -- i) {
            int cur = (x >> i & 1), need = (cur ^ 1);
            if (!nxt[rt][need]) ans = (ans << 1), need = !need;
            else ans = (ans << 1) | 1;
            rt = nxt[rt][need];
        }
        return ans;
    }
}
\end{minted}

\hypertarget{b.normal-trie}{%
\subsubsection{B.Normal Trie}\label{b.normal-trie}}

给定 \(n\) 个模式串 \(s_1, s_2, \dots, s_n\) 和 \(q\)
次询问,每次询问给定一个文本串 \(t_i\),请回答 \(s_1 \sim s_n\)
中有多少个字符串 \(s_j\) 满足 \(t_i\) 是 \(s_j\) 的\textbf{前缀}。保证
\(1 \leq T, n, q\leq 10^5\),且输入字符串的总长度不超过
\(3 \times 10^6\)。

\begin{minted}[fontsize=\footnotesize,breaklines,linenos]{cpp}
const int N = 3e6 + 10, MOD = 1e9 + 7;
const int DICT_SIZE = 65;
namespace Trie{
    int tot, nxt[N][DICT_SIZE], cnt[N];

    void clear(){
        for(int i = 0; i <= tot; i++){
            cnt[i] = 0;
            for(int j = 0; j <= 122; j++) nxt[i][j] = 0;
        }
        tot = 0;     
    }

    int getnum(char x){
        if(x >= 'A' && x <= 'Z') return x - 'A';
        else if(x >= 'a' && x <= 'z') return x - 'a' + 26;
        else return x - '0' + 52;
    } 

    void insert(char str[]){
        int p = 0, len = strlen(str);
        for(int i = 0; i < len; i++){
            int c = getnum(str[i]);
            if(!nxt[p][c]) nxt[p][c] = ++tot;
            p = nxt[p][c];
            cnt[p]++;
        }
    }

    int find(char str[]){
        int p = 0, len = strlen(str);
        for(int i = 0; i < len; i++){
            int c = getnum(str[i]);
            if(!nxt[p][c]) return 0;
            p = nxt[p][c];
        }
        return cnt[p];
    }
}

char s[N];

inline void solve(){
    Trie::clear();
    int n, q; cin >> n >> q;
    for(int i = 0; i < n; i++){
        cin >> s; Trie::insert(s);
    }
    for(int i = 0; i < q; i++){
        cin >> s; cout << Trie::find(s) << endl;
    }
}
/* TEST CASE
//INPUT
3
3 3
fusufusu
fusu
anguei
fusu
anguei
kkksc
5 2
fusu
Fusu
AFakeFusu
afakefusu
fusuisnotfake
Fusu
fusu
1 1
998244353
9
//OUTPUT
2
1
0
1
2
1
\end{minted}

\hypertarget{ux7b1bux5361ux5c14ux6811}{%
\subsection{10.笛卡尔树}\label{ux7b1bux5361ux5c14ux6811}}

给定一个 \(1 \sim n\) 的排列 \(p\),构建其笛卡尔树。

即构建一棵二叉树,满足:

\begin{enumerate}
\def\labelenumi{\arabic{enumi}.}
\tightlist
\item
  每个节点的编号满足二叉搜索树的性质。
\item
  节点 \(i\) 的权值为 \(p_i\),每个节点的权值满足小根堆的性质。
\end{enumerate}

这棵树的每个结点有两个子树,分为左右子树,子树可以为空;

一个结点的左子树中的所有结点的第一个权值都小于其第一个权值(空子树也满足);

一个结点的右子树中的所有结点的第一个权值都大于其第一个权值(空子树也满足);

一个结点的两棵子树中的所有结点的第二个权值都大于其第二个权值(空子树也满足)。

\begin{minted}[fontsize=\footnotesize,breaklines,linenos]{cpp}
const int N = 1e7 + 7;
int n, a[N], stk[N], ls[N], rs[N];

void build(int n){
    for (int i = 1, pos = 0, top = 0; i <= n; ++i){ //这是按下标顺序插入元素的代码
        pos = top;
        while (pos && a[stk[pos]] > a[i]) pos--;
        if (pos) rs[stk[pos]] = i;
        if (pos < top) ls[i] = stk[pos + 1];
        stk[top = ++pos] = i;
    }
}

inline void solve(){
    int n = read();
    for(int i = 1; i <= n; i++) a[i] = read();
    build(n);
    long long L = 0, R = 0;
    for (int i = 1; i <= n; ++i)
        L ^= 1LL * i * (ls[i] + 1), R ^= 1LL * i * (rs[i] + 1);
    printf("%lld %lld", L, R);
}
/*TEST CASE
//INPUT
5
4 1 3 2 5
//OUTPUT
19 21
*/
\end{minted}

\hypertarget{treap}{%
\subsection{11.Treap}\label{treap}}

FROM ECNU 板子库

\hypertarget{ux539fux59cbtreap}{%
\subsubsection{原始Treap}\label{ux539fux59cbtreap}}

\begin{itemize}
\tightlist
\item
  非旋 Treap
\item
  v 小根堆
\item
  lower 第一个大于等于的是第几个 (0-based)
\item
  upper 第一个大于的是第几个 (0-based)
\item
  split 左侧分割出 rk 个元素
\end{itemize}

\begin{minted}[fontsize=\footnotesize,breaklines,linenos]{cpp}
namespace treap {
    const int M = maxn * 17;
    extern struct P* const null;
    struct P {
        P *ls, *rs;
        int v, sz;
        unsigned rd;
        P(int v): ls(null), rs(null), v(v), sz(1), rd(rnd()) {}
        P(): sz(0) {}

        P* up() { sz = ls->sz + rs->sz + 1; return this; }
        int lower(int v) {
            if (this == null) return 0;
            return this->v >= v ? ls->lower(v) : rs->lower(v) + ls->sz + 1;
        }
        int upper(int v) {
            if (this == null) return 0;
            return this->v > v ? ls->upper(v) : rs->upper(v) + ls->sz + 1;
        }
    } *const null = new P, pool[M], *pit = pool;

    P* merge(P* l, P* r) {
        if (l == null) return r; if (r == null) return l;
        if (l->rd < r->rd) { l->rs = merge(l->rs, r); return l->up(); }
        else { r->ls = merge(l, r->ls); return r->up(); }
    }

    void split(P* o, int rk, P*& l, P*& r) {
        if (o == null) { l = r = null; return; }
        if (o->ls->sz >= rk) { split(o->ls, rk, l, o->ls); r = o->up(); }
        else { split(o->rs, rk - o->ls->sz - 1, o->rs, r); l = o->up(); }
    }
}
\end{minted}

\begin{itemize}
\tightlist
\item
  持久化 Treap
\end{itemize}

\begin{minted}[fontsize=\footnotesize,breaklines,linenos]{cpp}
namespace treap {
    const int M = maxn * 17 * 12;
    extern struct P* const null, *pit;
    struct P {
        P *ls, *rs;
        int v, sz;
        LL sum;
        P(P* ls, P* rs, int v): ls(ls), rs(rs), v(v), sz(ls->sz + rs->sz + 1),
                                                     sum(ls->sum + rs->sum + v) {}
        P() {}

        void* operator new(size_t _) { return pit++; }
        template<typename T>
        int rk(int v, T&& cmp) {
            if (this == null) return 0;
            return cmp(this->v, v) ? ls->rk(v, cmp) : rs->rk(v, cmp) + ls->sz + 1;
        }
        int lower(int v) { return rk(v, greater_equal<int>()); }
        int upper(int v) { return rk(v, greater<int>()); }
    } pool[M], *pit = pool, *const null = new P;
    P* merge(P* l, P* r) {
        if (l == null) return r; if (r == null) return l;
        if (rnd() % (l->sz + r->sz) < l->sz) return new P{l->ls, merge(l->rs, r), l->v};
        else return new P{merge(l, r->ls), r->rs, r->v};
    }
    void split(P* o, int rk, P*& l, P*& r) {
        if (o == null) { l = r = null; return; }
        if (o->ls->sz >= rk) { split(o->ls, rk, l, r); r = new P{r, o->rs, o->v}; }
        else { split(o->rs, rk - o->ls->sz - 1, l, r); l = new P{o->ls, l, o->v}; }
    }
}
\end{minted}

\begin{itemize}
\tightlist
\item
  带 pushdown 的持久化 Treap
\item
  注意任何修改操作前一定要 FIX
\end{itemize}

\begin{minted}[fontsize=\footnotesize,breaklines,linenos]{cpp}
int now;
namespace Treap {
    const int M = 10000000;
    extern struct P* const null, *pit;
    struct P {
        P *ls, *rs;
        int sz, time;
        LL cnt, sc, pos, add;
        bool rev;

        P* up() { sz = ls->sz + rs->sz + 1; sc = ls->sc + rs->sc + cnt; return this; } // MOD
        P* check() {
            if (time == now) return this;
            P* t = new(pit++) P; *t = *this; t->time = now; return t;
        };
        P* _do_rev() { rev ^= 1; add *= -1; pos *= -1; swap(ls, rs); return this; } // MOD
        P* _do_add(LL v) { add += v; pos += v; return this; } // MOD
        P* do_rev() { if (this == null) return this; return check()->_do_rev(); } // FIX & MOD
        P* do_add(LL v) { if (this == null) return this; return check()->_do_add(v); } // FIX & MOD
        P* _down() { // MOD
            if (rev) { ls = ls->do_rev(); rs = rs->do_rev(); rev = 0; }
            if (add) { ls = ls->do_add(add); rs = rs->do_add(add); add = 0; }
            return this;
        }
        P* down() { return check()->_down(); } // FIX & MOD
        void _split(LL p, P*& l, P*& r) { // MOD
            if (pos >= p) { ls->split(p, l, r); ls = r; r = up(); }
            else          { rs->split(p, l, r); rs = l; l = up(); }
        }
        void split(LL p, P*& l, P*& r) { // FIX & MOD
            if (this == null) l = r = null;
            else down()->_split(p, l, r);
        }
    } pool[M], *pit = pool, *const null = new P;
    P* merge(P* a, P* b) {
        if (a == null) return b; if (b == null) return a;
        if (rand() % (a->sz + b->sz) < a->sz) { a = a->down(); a->rs = merge(a->rs, b); return a->up(); }
        else                                 { b = b->down(); b->ls = merge(a, b->ls); return b->up(); }
    }
}
\end{minted}

\hypertarget{treap-ux5e8fux5217}{%
\subsubsection{Treap-序列}\label{treap-ux5e8fux5217}}

\begin{itemize}
\tightlist
\item
  区间 ADD,SUM
\end{itemize}

\begin{minted}[fontsize=\footnotesize,breaklines,linenos]{cpp}
namespace treap {
    const int M = 8E5 + 100;
    extern struct P*const null;
    struct P {
        P *ls, *rs;
        int sz, val, add, sum;
        P(int v, P* ls = null, P* rs = null): ls(ls), rs(rs), sz(1), val(v), add(0), sum(v) {}
        P(): sz(0), val(0), add(0), sum(0) {}

        P* up() {
            assert(this != null);
            sz = ls->sz + rs->sz + 1;
            sum = ls->sum + rs->sum + val + add * sz;
            return this;
        }
        void upd(int v) {
            if (this == null) return;
            add += v;
            sum += sz * v;
        }
        P* down() {
            if (add) {
                ls->upd(add); rs->upd(add);
                val += add;
                add = 0;
            }
            return this;
        }

        P* select(int rk) {
            if (rk == ls->sz + 1) return this;
            return ls->sz >= rk ? ls->select(rk) : rs->select(rk - ls->sz - 1);
        }
    } pool[M], *pit = pool, *const null = new P, *rt = null;

    P* merge(P* a, P* b) {
        if (a == null) return b->up();
        if (b == null) return a->up();
        if (rand() % (a->sz + b->sz) < a->sz) {
            a->down()->rs = merge(a->rs, b);
            return a->up();
        } else {
            b->down()->ls = merge(a, b->ls);
            return b->up();
        }
    }

    void split(P* o, int rk, P*& l, P*& r) {
        if (o == null) { l = r = null; return; }
        o->down();
        if (o->ls->sz >= rk) {
            split(o->ls, rk, l, o->ls);
            r = o->up();
        } else {
            split(o->rs, rk - o->ls->sz - 1, o->rs, r);
            l = o->up();
        }
    }

    inline void insert(int k, int v) {
        P *l, *r;
        split(rt, k - 1, l, r);
        rt = merge(merge(l, new (pit++) P(v)), r);
    }

    inline void erase(int k) {
        P *l, *r, *_, *t;
        split(rt, k - 1, l, t);
        split(t, 1, _, r);
        rt = merge(l, r);
    }

    P* build(int l, int r, int* a) {
        if (l > r) return null;
        if (l == r) return new(pit++) P(a[l]);
        int m = (l + r) / 2;
        return (new(pit++) P(a[m], build(l, m - 1, a), build(m + 1, r, a)))->up();
    }
};
\end{minted}

\begin{itemize}
\tightlist
\item
  区间 REVERSE,ADD,MIN
\end{itemize}

\begin{minted}[fontsize=\footnotesize,breaklines,linenos]{cpp}
namespace treap {
    extern struct P*const null;
    struct P {
        P *ls, *rs;
        int sz, v, add, m;
        bool flip;
        P(int v, P* ls = null, P* rs = null): ls(ls), rs(rs), sz(1), v(v), add(0), m(v), flip(0) {}
        P(): sz(0), v(INF), m(INF) {}

        void upd(int v) {
            if (this == null) return;
            add += v; m += v;
        }
        void rev() {
            if (this == null) return;
            swap(ls, rs);
            flip ^= 1;
        }
        P* up() {
            assert(this != null);
            sz = ls->sz + rs->sz + 1;
            m = min(min(ls->m, rs->m), v) + add;
            return this;
        }
        P* down() {
            if (add) {
                ls->upd(add); rs->upd(add);
                v += add;
                add = 0;
            }
            if (flip) {
                ls->rev(); rs->rev();
                flip = 0;
            }
            return this;
        }

        P* select(int k) {
            if (ls->sz + 1 == k) return this;
            if (ls->sz >= k) return ls->select(k);
            return rs->select(k - ls->sz - 1);
        }

    } pool[M], *const null = new P, *pit = pool, *rt = null;

    P* merge(P* a, P* b) {
        if (a == null) return b;
        if (b == null) return a;
        if (rnd() % (a->sz + b->sz) < a->sz) {
            a->down()->rs = merge(a->rs, b);
            return a->up();
        } else {
            b->down()->ls = merge(a, b->ls);
            return b->up();
        }
    }

    void split(P* o, int k, P*& l, P*& r) {
        if (o == null) { l = r = null; return; }
        o->down();
        if (o->ls->sz >= k) {
            split(o->ls, k, l, o->ls);
            r = o->up();
        } else {
            split(o->rs, k - o->ls->sz - 1, o->rs, r);
            l = o->up();
        }
    }

    P* build(int l, int r, int* v) {
        if (l > r) return null;
        int m = (l + r) >> 1;
        return (new (pit++) P(v[m], build(l, m - 1, v), build(m + 1, r, v)))->up();
    }

    void go(int x, int y, void f(P*&)) {
        P *l, *m, *r;
        split(rt, y, l, r);
        split(l, x - 1, l, m);
        f(m);
        rt = merge(merge(l, m), r);
    }
}
using namespace treap;
int a[maxn], n, x, y, Q, v, k, d;
char s[100];

int main() {
    cin >> n;
    FOR (i, 1, n + 1) scanf("%d", &a[i]);
    rt = build(1, n, a);
    cin >> Q;
    while (Q--) {
        scanf("%s", s);
        if (s[0] == 'A') {
            scanf("%d%d%d", &x, &y, &v);
            go(x, y, [](P*& o){ o->upd(v); });
        } else if (s[0] == 'R' && s[3] == 'E') {
            scanf("%d%d", &x, &y);
            go(x, y, [](P*& o){ o->rev(); });
        } else if (s[0] == 'R' && s[3] == 'O') {
            scanf("%d%d%d", &x, &y, &d);
            d %= y - x + 1;
            go(x, y, [](P*& o){
                P *l, *r;
                split(o, o->sz - d, l, r);
                o = merge(r, l);
            });
        } else if (s[0] == 'I') {
            scanf("%d%d", &k, &v);
            go(k + 1, k, [](P*& o){ o = new (pit++) P(v); });
        } else if (s[0] == 'D') {
            scanf("%d", &k);
            go(k, k, [](P*& o){ o = null; });
        } else if (s[0] == 'M') {
            scanf("%d%d", &x, &y);
            go(x, y, [](P*& o) {
                printf("%d\n", o->m);
            });
        }
    }
}

\end{minted}

\begin{itemize}
\tightlist
\item
  持久化
\end{itemize}

\begin{minted}[fontsize=\footnotesize,breaklines,linenos]{cpp}
namespace treap {
    struct P;
    extern P*const null;
    P* N(P* ls, P* rs, LL v, bool fill);
    struct P {
        P *const ls, *const rs;
        const int sz, v;
        const LL sum;
        bool fill;
        int cnt;

        void split(int k, P*& l, P*& r) {
            if (this == null) { l = r = null; return; }
            if (ls->sz >= k) {
                ls->split(k, l, r);
                r = N(r, rs, v, fill);
            } else {
                rs->split(k - ls->sz - fill, l, r);
                l = N(ls, l, v, fill);
            }
        }


    } *const null = new P{0, 0, 0, 0, 0, 0, 1};

    P* N(P* ls, P* rs, LL v, bool fill) {
        ls->cnt++; rs->cnt++;
        return new P{ls, rs, ls->sz + rs->sz + fill, v, ls->sum + rs->sum + v, fill, 1};
    }

    P* merge(P* a, P* b) {
        if (a == null) return b;
        if (b == null) return a;
        if (rand() % (a->sz + b->sz) < a->sz)
            return N(a->ls, merge(a->rs, b), a->v, a->fill);
        else
            return N(merge(a, b->ls), b->rs, b->v, b->fill);
    }

    void go(P* o, int x, int y, P*& l, P*& m, P*& r) {
        o->split(y, l, r);
        l->split(x - 1, l, m);
    }
}
\end{minted}

\hypertarget{section}{%
\subsection{}\label{section}}

\hypertarget{ux83abux961f}{%
\subsection{12.莫队}\label{ux83abux961f}}

\hypertarget{ux6269ux5c55ux987aux5e8f}{%
\subsubsection{扩展顺序}\label{ux6269ux5c55ux987aux5e8f}}

\begin{minted}[fontsize=\footnotesize,breaklines,linenos]{cpp}
while (l > q.l) mv(--l, 1);
while (r < q.r) mv(r++, 1);
while (l < q.l) mv(l++, -1);
while (r > q.r) mv(--r, -1);
\end{minted}

\hypertarget{cdq-ux5206ux6cbb}{%
\subsection{13.CDQ 分治}\label{cdq-ux5206ux6cbb}}

\begin{minted}[fontsize=\footnotesize,breaklines,linenos]{cpp}
const int maxn = 2E5 + 100;
struct P {
    int x, y;
    int* f;
    bool d1, d2;
} a[maxn], b[maxn], c[maxn];
int f[maxn];

void go2(int l, int r) {
    if (l + 1 == r) return;
    int m = (l + r) >> 1;
    go2(l, m); go2(m, r);
    for(int i = l; i < m; i++) b[i].d2 = 0;
    for(int i = m; i < r; i++) b[i].d2 = 1;
    merge(b + l, b + m, b + m, b + r, c + l, [](const P& a, const P& b)->bool {
            if (a.y != b.y) return a.y < b.y;
            return a.d2 > b.d2;
        });
    int mx = -1;
    for(int i = l; i < r; i++) {
        if (c[i].d1 && c[i].d2) *c[i].f = max(*c[i].f, mx + 1);
        if (!c[i].d1 && !c[i].d2) mx = max(mx, *c[i].f);
    }
    for(int i = l; i < r; i++)  b[i] = c[i];
}

void go1(int l, int r) { // [l, r)
    if (l + 1 == r) return;
    int m = (l + r) >> 1;
    go1(l, m);
    for(int i = l; i < m; i++) a[i].d1 = 0;
    for(int i = m; i < r; i++) a[i].d1 = 1;
    copy(a + l, a + r, b + l);
    sort(b + l, b + r, [](const P& a, const P& b)->bool {
            if (a.x != b.x) return a.x < b.x;
            return a.d1 > b.d1;
        });
    go2(l, r);
    go1(m, r);
}

\end{minted}

\hypertarget{ux73c2ux6735ux8389ux6811ux8001ux53f8ux673aux6811}{%
\subsection{14.珂朵莉树/老司机树}\label{ux73c2ux6735ux8389ux6811ux8001ux53f8ux673aux6811}}

\begin{minted}[fontsize=\footnotesize,breaklines,linenos]{cpp}
struct node{
    int l, r;
    mutable int val;
    node (int lpos): l(lpos) {}
    node (int lpos, int rpos, int vall): l(lpos), r(rpos), val(vall) {}
    bool operator< (const node &a) const { return l < a.l; }
};

set<node> s;
using sit = set<node>::iterator;

sit split(int pos){
    sit it = s.lower_bound(node(pos));
    if(it != s.end() && it -> l == pos) return it;
    --it;
    int l = it -> l, r = it -> r, val = it -> val;
    s.erase(it);
    s.insert(node(l, pos - 1, val));
    return s.insert(node(pos, r, val)).first;
}

void assign(int l, int r, int val){
    sit itr = split(r + 1), itl = split(l);
    s.erase(itl, itr);
    s.insert(node(l, r, val));
}

void add(int l, int r, int val){
    sit itr = split(r + 1), itl = split(l);
    //for(auto it = itl; it != itr; it++) it -> val += val;
    while(itl != itr) itl -> val += val, itl++;
}

int kth(int l, int r, int k){
    sit itr = split(r + 1), itl = split(l);
    vector<pair<int, int>> v;
    v.clear();
    for(sit it = itl; it != itr; it++) v.emplace_back(make_pair(it -> val, it -> r - it -> l + 1));
    sort(v.begin(), v.end());
    for(int i = 0; i < v.size(); i++){
        k -= v[i].second;
        if(k <= 0) return v[i].first;
    }
    return -1;
}

int binpow(int x, int y, int mod, int res = 1){
    for (x %= mod; y; y >>= 1, (x *= x) %= mod) if (y & 1) (res *= x) %= mod;
    return res;
}

int query(int l, int r, int x, int y){
    sit itr = split(r + 1), itl = split(l);
    int res(0);
    for(sit it = itl; it != itr; it++)
        res = (res + (it -> r - it -> l + 1) * binpow(it -> val, x, y)) % y;
    return res;
}

int n, m, vmax, seed;
int rnd() {
    int ret = (int)seed;
    seed = (seed * 7 + 13) % 1000000007;
    return ret;
}

inline void solve(){
    cin >> n >> m >> seed >> vmax;
    for(int i = 1; i <= n; i++){
        int a = rnd() % vmax + 1;
        s.insert(node{i, i, (int) a});
    }
    s.insert(node{n + 1, n + 1, 0});
    for(int i = 1; i <= m; i++){
        int l, r, x, y;
        int op = rnd() % 4 + 1;
        l = rnd() % n + 1, r = rnd() % n + 1;
        if(l > r) swap(l, r);
        if(op == 3) x = rnd() % (r - l + 1) + 1;
        else x = rnd() % vmax + 1;
        if(op == 4) y = rnd() % vmax + 1;
        if(op == 1) add(l, r, x);
        else if(op == 2) assign(l, r, x);
        else if(op == 3) cout << kth(l, r, x) << endl;
        else if(op == 4) cout << query(l, r, x, y) << endl;
    }
}
\end{minted}

\hypertarget{exstl}{%
\subsection{exSTL}\label{exstl}}

\hypertarget{ux4f18ux5148ux961fux5217}{%
\subsubsection{优先队列}\label{ux4f18ux5148ux961fux5217}}

\begin{itemize}
\tightlist
\item
  binary\_heap\_tag
\item
  pairing\_heap\_tag 支持修改
\item
  thin\_heap\_tag 如果修改只有 increase 则较快,不支持 join
\end{itemize}

\begin{minted}[fontsize=\footnotesize,breaklines,linenos]{cpp}
#include<ext/pb_ds/priority_queue.hpp>
using namespace __gnu_pbds;

typedef __gnu_pbds::priority_queue<LL, less<LL>, pairing_heap_tag> PQ;
__gnu_pbds::priority_queue<int, cmp, pairing_heap_tag>::point_iterator it;
PQ pq, pq2;

int main() {
    auto it = pq.push(2);
    pq.push(3);
    assert(pq.top() == 3);
    pq.modify(it, 4);
    assert(pq.top() == 4);
    pq2.push(5);
    pq.join(pq2);
    assert(pq.top() == 5);
}
\end{minted}

\hypertarget{ux5e73ux8861ux6811}{%
\subsubsection{平衡树}\label{ux5e73ux8861ux6811}}

\begin{itemize}
\item
  ov\_tree\_tag
\item
  rb\_tree\_tag
\item
  splay\_tree\_tag
\item
  mapped: null\_type 或 null\_mapped\_type(旧版本) 为空
\item
  Node\_Update 为 tree\_order\_statistics\_node\_update 时才可以
  find\_by\_order \& order\_of\_key
\item
  find\_by\_order 找 order + 1 小的元素 (其实都是从 0
  开始计数),或者有 order 个元素比它小的 key
\item
  order\_of\_key 有多少个比 r\_key 小的元素
\item
  join \& split
\end{itemize}

\begin{minted}[fontsize=\footnotesize,breaklines,linenos]{cpp}
#include <ext/pb_ds/assoc_container.hpp>
using namespace __gnu_pbds;
using Tree = tree<int, null_type, less<int>, rb_tree_tag, tree_order_statistics_node_update>;
Tree t;
\end{minted}

\hypertarget{ux6301ux4e45ux5316ux5e73ux8861ux6811}{%
\subsubsection{持久化平衡树}\label{ux6301ux4e45ux5316ux5e73ux8861ux6811}}

\begin{minted}[fontsize=\footnotesize,breaklines,linenos]{cpp}
#include <ext/rope>
using namespace __gnu_cxx;
rope<int> s;

int main() {
    FOR (i, 0, 5) s.push_back(i); // 0 1 2 3 4
    s.replace(1, 2, s); // 0 (0 1 2 3 4) 3 4
    auto ss = s.substr(2, 2); // 1 2、
    s.erase(2, 2); // 0 1 4
    s.insert(2, s); // equal to s.replace(2, 0, s)
    assert(s[2] == s.at(2)); // 2
}
\end{minted}

\hypertarget{ux54c8ux5e0cux8868}{%
\subsubsection{哈希表}\label{ux54c8ux5e0cux8868}}

\begin{minted}[fontsize=\footnotesize,breaklines,linenos]{cpp}
#include<ext/pb_ds/assoc_container.hpp>
#include<ext/pb_ds/hash_policy.hpp>
using namespace __gnu_pbds;

gp_hash_table<int, int> mp;
cc_hash_table<int, int> mp;
\end{minted}

\hypertarget{zkwux7ebfux6bb5ux6811ux4e0dux5e26ux4feermqonlog-log-no1}{%
\subsection{\texorpdfstring{ZKW线段树+不带修RMQ(\(O(N\log \log N)+O(1)\))}{ZKW线段树+不带修RMQ(O(N\textbackslash log \textbackslash log N)+O(1))}}\label{zkwux7ebfux6bb5ux6811ux4e0dux5e26ux4feermqonlog-log-no1}}

普通的 zkw 线段树解法分为线性建树和单次 \(O(logn)\)
的查询,显然查询是瓶颈。

观察 zkw 线段树的查询结构:

\begin{minted}[fontsize=\footnotesize,breaklines,linenos]{cpp}
  for (l += M - 1, r += M + 1; l ^ r ^ 1; l >>= 1, r >>= 1) {
    if (~l & 1) ret = max(ret, zkw[l ^ 1]);
    if (r & 1) ret = max(ret, zkw[r ^ 1]);
  }
\end{minted}

发现就是两边链查询,查询的信息类似于一条链上所有左/右儿子的兄弟们的最大值。

它是可以快速合并的,直接采用倍增优化(树上 ST 表),然后 \(O(1)\)
回答询问。

由于树高 \(O(logn)\),故预处理 ST 的时空复杂度均为 \(O(nloglogn)\)。

甚至可以原线段树都不要了,直接用数组建出 ST 表(见代码)。

综上,预处理时间复杂度:\(O(n\log \log n)\),单次查询复杂度
\(O(1)\),空间复杂度\(O(n\log⁡ \log⁡n)\),优点是代码较短。

例题:第一行输入序列长度和查询数,接着一行输入序列,接着若干行每行一个查询区间,所有输入为
\(10^5\) 以内的正整数。

\begin{minted}[fontsize=\footnotesize,breaklines,linenos]{cpp}
#include <bits/stdc++.h>
using namespace std;
const int N = 1e5 + 9, Mx = 1 << 17;
int n, m, M = 1, lg[Mx << 1];
pair<int, int> val[Mx << 1][5];
int main() {
  scanf("%d%d", &n, &m);
  while (M <= n) M <<= 1;
  for (int i = M + 1, a; i <= M + n; ++i) {
    scanf("%d", &a);
    ((i & 1) ? val[i ^ 1][0].first : val[i ^ 1][0].second) = a;
  }
  for (int i = M - 1; i > 1; --i)
    ((i & 1) ? val[i ^ 1][0].first : val[i ^ 1][0].second) =
        max(val[i << 1][0].first, val[i << 1 | 1][0].second);
  for (int i = 2; i < M + M; ++i) lg[i] = lg[i >> 1] + 1;
  for (int i = 2; i < M + M; ++i)
    for (int h = 1; h < 5; ++h) {
      val[i][h].first =
          max(val[i][h - 1].first, val[i >> (1 << (h - 1))][h - 1].first);
      val[i][h].second =
          max(val[i][h - 1].second, val[i >> (1 << (h - 1))][h - 1].second);
    }
  for (int l, r, len; m; --m) {
    scanf("%d%d", &l, &r);
    l += M - 1, r += M + 1, len = lg[l ^ r];
    printf("%d\n", max(max(val[l][lg[len]].first,
                           val[l >> (len - (1 << lg[len]))][lg[len]].first),
                       max(val[r][lg[len]].second,
                           val[r >> (len - (1 << lg[len]))][lg[len]].second)));
  }
  return 0;
}
\end{minted}

\end{document}
