\PassOptionsToPackage{unicode=true}{hyperref} % options for packages loaded elsewhere
\PassOptionsToPackage{hyphens}{url}
%
\documentclass[]{article}
\usepackage{lmodern}
\usepackage{amssymb,amsmath}

% settings
\usepackage{minted}
%



\usepackage{ifxetex,ifluatex}
\usepackage{fixltx2e} % provides \textsubscript
\ifnum 0\ifxetex 1\fi\ifluatex 1\fi=0 % if pdftex
  \usepackage[T1]{fontenc}
  \usepackage[utf8]{inputenc}
  \usepackage{textcomp} % provides euro and other symbols
\else % if luatex or xelatex
  \usepackage{unicode-math}
  \defaultfontfeatures{Ligatures=TeX,Scale=MatchLowercase}
    \setmainfont[]{Source Han Serif CN}
    \setsansfont[]{Source Han Sans CN}
    \setmonofont[Mapping=tex-ansi]{Source Code Pro}
  \ifxetex
    \usepackage{xeCJK}
    \setCJKmainfont[]{Source Han Serif CN}
  \fi
  \ifluatex
    \usepackage[]{luatexja-fontspec}
    \setmainjfont[]{Source Han Serif CN}
  \fi
\fi
% use upquote if available, for straight quotes in verbatim environments
\IfFileExists{upquote.sty}{\usepackage{upquote}}{}
% use microtype if available
\IfFileExists{microtype.sty}{%
\usepackage[]{microtype}
\UseMicrotypeSet[protrusion]{basicmath} % disable protrusion for tt fonts
}{}
\IfFileExists{parskip.sty}{%
\usepackage{parskip}
}{% else
\setlength{\parindent}{0pt}
\setlength{\parskip}{6pt plus 2pt minus 1pt}
}
\usepackage{hyperref}
\hypersetup{
            pdfborder={0 0 0},
            breaklinks=true}
\urlstyle{same}  % don't use monospace font for urls
\usepackage[margin=2cm]{geometry}
\setlength{\emergencystretch}{3em}  % prevent overfull lines
\providecommand{\tightlist}{%
  \setlength{\itemsep}{0pt}\setlength{\parskip}{0pt}}
\setcounter{secnumdepth}{0}
% Redefines (sub)paragraphs to behave more like sections
\ifx\paragraph\undefined\else
\let\oldparagraph\paragraph
\renewcommand{\paragraph}[1]{\oldparagraph{#1}\mbox{}}
\fi
\ifx\subparagraph\undefined\else
\let\oldsubparagraph\subparagraph
\renewcommand{\subparagraph}[1]{\oldsubparagraph{#1}\mbox{}}
\fi


% set default figure placement to htbp
\makeatletter
\def\fps@figure{htbp}
\makeatother

\usepackage{minted}



\date{}

\title{\vspace{50mm} \huge Standard Code Library \\[20pt]}
\author{HeartFireY, eroengine, yezzz \\[10pt] Jiangxi Normal University}
\date{\today}


\begin{document}

\begin{titlepage}

\maketitle

\end{titlepage}

\newpage

\renewcommand\labelitemi{$\bullet$}

{
\setcounter{tocdepth}{3}
\tableofcontents
\newpage
}








\hypertarget{ux56feux8bba}{%
\section{图论}\label{ux56feux8bba}}

\hypertarget{lca}{%
\subsection{1.LCA}\label{lca}}

\hypertarget{ux500dux589e}{%
\subsubsection{倍增}\label{ux500dux589e}}

\begin{minted}[fontsize=\footnotesize,breaklines,linenos]{cpp}
int n;
vector<int>to[N];
int rt;
int depth[N];
int fa[N][21];//fa[u][i] 表示 u 开始向上 2^i 的父节点
void dfs_lca(int u, int father) {
    depth[u] = depth[father] + 1;
    fa[u][0] = father;
    for (int i = 1; (1 << i) <= depth[u]; i++) {
        fa[u][i] = fa[fa[u][i - 1]][i - 1];
    }
    for (auto ed : to[u]) {
        int v = ed;
        if (v != father) {
            dfs_lca(v, u);
        }
    }
}
int lca(int u,int v) {
    if (depth[u] < depth[v])swap(u, v);
    for (int k = 20; k >= 0; k--) {
        if (depth[fa[u][k]] >= depth[v]) {
            u = fa[u][k];
        }
    }
    if (u == v) {
        return u;
    }
    for (int k = 20; k >= 0; k--) {
        if (fa[u][k] != fa[v][k]) {
            u = fa[u][k];
            v = fa[v][k];
        }
    }
    return fa[u][0];
}
void solve() {
    cin >> n;
    for (int i = 1; i <= n; i++) {
        int u, v;
        cin >> u >> v;
        if (v == -1) {
            rt = u;
        }
        to[u].pb(v);
        to[v].pb(u);
    }
    dfs_lca(rt, 0);
    int q;
    cin >> q;
    while (q--) {
        int u, v;
        cin >> u >> v;
        cout << lca(u, v) << "\n";
    }
}

\end{minted}

\hypertarget{hld-ux6811ux5256}{%
\subsubsection{HLD 树剖}\label{hld-ux6811ux5256}}

\begin{minted}[fontsize=\footnotesize,breaklines,linenos]{cpp}
#include <bits/stdc++.h>

#define maxm 200010
namespace LCA{
    struct edge{ int to, len, next; } E[maxm];
    int cnt, last[maxm], fa[maxm], top[maxm], deep[maxm], siz[maxm], son[maxm], val[maxm];
    void addedge(int a, int b, int len = 0){
        E[++cnt] = (edge){b, len, last[a]}, last[a] = cnt;
    }
    void dfs1(int x){
        deep[x] = deep[fa[x]] + 1;
        siz[x] = 1;
        for (int i = last[x]; i; i = E[i].next){
            int to = E[i].to;
            if (fa[x] != to && !fa[to]){
                val[to] = E[i].len;
                fa[to] = x;
                dfs1(to);
                siz[x] += siz[to];
                if (siz[son[x]] < siz[to]) son[x] = to;
            }
        }
    }
    void dfs2(int x){
        if (x == son[fa[x]]) top[x] = top[fa[x]];
        else top[x] = x;
        for (int i = last[x]; i; i = E[i].next)
            if (fa[E[i].to] == x) dfs2(E[i].to);
    }
    void init(int root) { dfs1(root), dfs2(root); }
    int query(int x, int y){
        for (; top[x] != top[y]; deep[top[x]] > deep[top[y]] ? x = fa[top[x]] : y = fa[top[y]]);
        return deep[x] < deep[y] ? x : y;
    }
}
int n, m, x, y, v;
int main(){
    scanf("%d%d", &n, &m);
    for (int i = 1; i < n; i++){
        scanf("%d%d", &x, &y);
        LCA::addedge(x, y, v);
        LCA::addedge(y, x, v);
    }
    LCA::init(1);
    for (int i = 1; i <= m; i++){
        scanf("%d%d", &x, &y);
        printf("%d\n", LCA::query(x, y));
    }
    return 0;
}
\end{minted}

\hypertarget{kruskalux91cdux6784ux6811}{%
\subsection{2.Kruskal重构树}\label{kruskalux91cdux6784ux6811}}

\begin{minted}[fontsize=\footnotesize,breaklines,linenos]{cpp}
const int N = 1e5 + 10;
int n = 0, m = 0;

namespace Graph
{
    struct edge { int to, nxt, val; } edges[N << 1];
    int cnt, head[N << 1], val[N << 1];
    void add(int u, int v, int val = 0){
        edges[++cnt] = (edge){v, head[u], val};
        head[u] = cnt;
    }
} // namespace Graph

namespace KR{
    using Graph::add;
    struct edge{ 
        int u, v, w;
        const bool operator< (const edge &x) const { return w < x.w; }
    } edges[N] ;
    int fa[N];
    void init(int n){ for(int i = 1; i <= n; i++) fa[i] = i; }
    int find(int x){ return fa[x] == x ? x : (fa[x] = find(fa[x])); }

    
    void kruskal(){
        int tot = 0, cnt = n;
        sort(edges + 1, edges + 1 + m);
        for(int i = 1; i <= m; i++){
            int fau = find(edges[i].u), fav = find(edges[i].v);
            if(fau != fav){
                cnt++, fa[fau] = fa[fav] = cnt;
                add(fau, cnt), add(cnt, fau);
                add(fav, cnt), add(cnt, fav);
                Graph::val[cnt] = edges[i].w;
                tot++;
            }
            if(tot == n - 1) break;
        }

    }
} // namespace KR
\end{minted}

\hypertarget{km-ux4e8cux5206ux56feux6700ux5927ux6743ux5339ux914d}{%
\subsection{3.KM
二分图最大权匹配}\label{km-ux4e8cux5206ux56feux6700ux5927ux6743ux5339ux914d}}

\begin{minted}[fontsize=\footnotesize,breaklines,linenos]{cpp}
/*好难懂啊,还是抄板子吧*/
typedef long long ll;
typedef unsigned long long ull;//卡精度
typedef pair<int,int> PII;

const int N = 507, M = 5e3 +7, maxn = 1007;
const int mod = 1e9+7;
const ll INF = 1e15+7;

double w[N][N];//边权
double la[N], lb[N];//左、右部点的顶标
bool va[N], vb[N];//访问标记,是否在交错树中
int match[N];//右部点匹配的左部点(一个只能匹配一个嘛)
int n;
double delta, upd[N];
int p[N];
double c[N];

void bfs(int x)
{
    int a, y = 0, y1 = 0;

    for(int i = 1; i <= n; ++ i)
        p[i] = 0, c[i] = INF;

    match[y] = x;

    do{
        a = match[y], delta = INF, vb[y] = true;
        for(int b = 1; b <= n; ++ b){
            if(!vb[b]){
                if(c[b] > la[a] + lb[b] - w[a][b])
                    c[b] = la[a] + lb[b] - w[a][b], p[b] = y;
                if(c[b] < delta)//Δ还是取最小的
                    delta = c[b], y1 = b;
            }
        }
        for(int b = 0; b <= n; ++ b)
            if(vb[b])
                la[match[b]] -= delta, lb[b] += delta;
            else c[b] -= delta;
        y = y1;
    }while(match[y]);
    while(y) match[y] = match[p[y]], y = p[y];
}

double KM()
{
    for(int i = 1; i <= n; ++ i)
        match[i] = lb[i] = la[i] = 0;
    for(int i = 1; i <= n; ++ i){
        for(int j = 1; j <= n; ++ j)
            vb[j] = false;
        bfs(i);
    }
    double res = 0;
    for(int y = 1; y <= n; ++ y)
        res += w[match[y]][y];
    return res;
}
\end{minted}

\hypertarget{ux4e8cux5206ux56feux5339ux914d}{%
\subsection{4.二分图匹配}\label{ux4e8cux5206ux56feux5339ux914d}}

\begin{minted}[fontsize=\footnotesize,breaklines,linenos]{cpp}
bool dfs(int u, int tag)
{
    if (vis[u] == tag) return false; // 即说明已经被访问了,就不需要清空了
    // 其实只是将bool改成用数字标记,快的也很有限
    vis[u] = tag;
    for (auto v : e[u])
        if(!pre[v] || dfs(pre[v], tag))
        {
            pre[v] = u;
            return true;
        }
    return false;
}
\end{minted}

\hypertarget{ux62d3ux6251ux6392ux5e8f}{%
\subsection{5.拓扑排序}\label{ux62d3ux6251ux6392ux5e8f}}

\begin{minted}[fontsize=\footnotesize,breaklines,linenos]{cpp}
const int N = 1e5 + 10;
std::vector<int> g[N];
int in_cnt[N];

bool TopoSort(int n){
    std::vector<int> ans;
    std::queue<int> q;
    for(int i = 1; i <= n; i++) if(!in_cnt[i]) q.emplace(i);
    while(q.size()){
        int now = q.front(); q.pop();
        ans.emplace_back(now);
        for(auto nxt : g[now]) if(!--in_cnt[nxt]) q.emplace(nxt);
    }
    if(ans.size() == n){
        for(auto ansi : ans) cout << ansi << ' ';
        return true;
    }
    else return false;
}

inline void solve(){
    int n, m; std::cin >> n >> m;
    for(int i = 1; i <= m; i++){
        int u, v; std::cin >> u >> v;
        g[u].emplace_back(v);
        in_cnt[v]++;
    }
    bool status = TopoSort(n);
}
\end{minted}

\hypertarget{ekspfa}{%
\subsection{6.EK+SPFA}\label{ekspfa}}

\begin{minted}[fontsize=\footnotesize,breaklines,linenos]{cpp}
#include <bits/stdc++.h>
using namespace std;

const int N = 5100, M = 20010, INF = 1e8;
int n, m, s, t;
template <int N,int M>struct MCMF{
    //EK+spfa
    const int INF=1e8;
    int S,T;
    MCMF(int s,int t):S(s),T(t){}
    int h[N], e[M], f[M], w[M], ne[M], idx;
    int q[N], d[N], pre[N], incf[N];
    bool st[N];

    void add(int a, int b, int c, int d)
    {
        e[idx] = b, f[idx] = c, w[idx] = d, ne[idx] = h[a], h[a] = idx ++ ;
        e[idx] = a, f[idx] = 0, w[idx] = -d, ne[idx] = h[b], h[b] = idx ++ ;
    }

    bool spfa()
    {
        int hh = 0, tt = 1;
        memset(d, 0x3f, sizeof d);
        memset(incf, 0, sizeof incf);
        q[0] = S, d[S] = 0, incf[S] = INF;
        while (hh != tt)
        {
            int t = q[hh ++ ];
            if (hh == N) hh = 0;
            st[t] = false;

            for (int i = h[t]; ~i; i = ne[i])
            {
                int ver = e[i];
                if (f[i] && d[ver] > d[t] + w[i])
                {
                    d[ver] = d[t] + w[i];
                    pre[ver] = i;
                    incf[ver] = min(f[i], incf[t]);
                    if (!st[ver])
                    {
                        q[tt ++ ] = ver;
                        if (tt == N) tt = 0;
                        st[ver] = true;
                    }
                }
            }
        }

        return incf[T] > 0;
    }

    void EK(int& flow, int& cost)
    {
        flow = cost = 0;
        while (spfa())
        {
            int t = incf[T];
            flow += t, cost += t * d[T];
            for (int i = T; i != S; i = e[pre[i] ^ 1])
            {
                f[pre[i]] -= t;
                f[pre[i] ^ 1] += t;
            }
        }
    }
};
signed main()
{
    int x;
    cin>>n;
    s=0,t=n+n+1;
    MCMF<N,M>mcmf1(s,t);
    MCMF<N,M>mcmf2(s,t);

    memset(mcmf1.h, -1, sizeof mcmf1.h);
    for(int i=1;i<=n;i++){
        for(int j=1;j<=n;j++){
            cin>>x;
            mcmf1.add(i,j+n,1,x);
            mcmf2.add(i,j+n,1,-x);
        }
        mcmf1.add(0,i,1,0);
        mcmf1.add(i+n,n+n+1,1,0);
        mcmf2.add(0,i,1,0);
        mcmf2.add(i+n,n+n+1,1,0);
    }

    int flow, cost,flow2,cost2;
    mcmf1.EK(flow, cost);
    //mcmf2.EK(flow2, cost2);
    printf("%d %d\n",cost,-cost2);

    return 0;
}
\end{minted}

\hypertarget{dinic}{%
\subsection{7.Dinic}\label{dinic}}

\begin{minted}[fontsize=\footnotesize,breaklines,linenos]{cpp}
#include <bits/stdc++.h>
using namespace std;
template <int N> struct Dinic {
    const int INF = 1e9;
    struct E {
        int to, cap, rev;
    };
    vector<E> G[N];
    int lev[N], cur[N];
    inline void add(int x, int y, int c) {
        G[x].push_back({ y, c, (int)G[y].size() });
        G[y].push_back({ x, 0, (int)G[x].size() - 1 });
    }
    void bfs(int s) {
        queue<int> q;
        memset(lev, -1, sizeof lev);

        for (lev[s] = 0, q.push(s); q.size();) {
            int x = q.front();
            q.pop();

            for (auto &e : G[x])
                if (e.cap > 0 && lev[e.to] < 0)
                    lev[e.to] = lev[x] + 1, q.push(e.to);
        }
    }
    int dfs(int x, int t, int f) {
        if (x == t)
            return f;

        for (int &i = cur[x], sz = G[x].size(), d; i < sz; i++) {
            auto &e = G[x][i];

            if (e.cap > 0 && lev[x] < lev[e.to]) {
                if ((d = dfs(e.to, t, min(f, e.cap))) > 0) {
                    e.cap -= d, G[e.to][e.rev].cap += d;
                    return d;
                }
            }
        }

        return 0;
    }
    int64_t maxflow(int s, int t) {
        for (int64_t flow = 0, f;;) {
            bfs(s);

            if (lev[t] < 0)
                return flow;

            memset(cur, 0, sizeof cur);

            while ((f = dfs(s, t, INF)) > 0)
                flow += f;
        }
    }
};
Dinic<1005> din;
signed main() {
    cin.tie(0)->sync_with_stdio(0);
    int n, m, s, t;
    cin >> n ;
    for(int i=1;i<=n;i++){
        for(int j=1;j<=n;j++){
            int x;
            cin>>x;
            din.add(0,i,x);
            din.add(j+100,201,x);
            din.add(i,j+100,1e9);
        }
    }
    return cout << din.maxflow(0, 201), 0;
}
\end{minted}

\hypertarget{mcmfdijek}{%
\subsection{8.MCMF(dij+ek)}\label{mcmfdijek}}

\begin{minted}[fontsize=\footnotesize,breaklines,linenos]{cpp}
#include <bits/stdc++.h>
using namespace std;
const int mod = 1000000007,N = 100005,inf = 1e9;
template <int N> struct MCMF{
    struct E{
        int to,cap,val,inv;
    };
    vector <E> g[N];
    int dis[N],now[N],h[N],pre[N],preu[N];
    void add(int u,int v,int f,int w){
        g[u].push_back({v,f,w,(int)g[v].size()});
        g[v].push_back({u,0,-w,(int)g[u].size()-1});
    }
    void dijkstra(int st){
        priority_queue<pair<int,int>,vector<pair<int,int>>,greater<pair<int,int>>>q;
        memset(dis,0x3f,sizeof dis);
        memset(pre,-1,sizeof pre);
        memset(preu,-1,sizeof preu);
        dis[st]=0;q.push({0,st});
        while(q.size()){
            auto [d,u]=q.top();q.pop();
            if(dis[u]<d)continue;
            int x=0;
            for(auto [v,f,w,inv]:g[u]){
                if(f&&dis[v]>dis[u]+w+h[u]-h[v]){
                    dis[v]=dis[u]+h[u]-h[v]+w;
                    pre[v]=x;
                    preu[v]=u;
                    q.push({dis[v],v});
                }
                x++;
            }
        }
    }
    pair<int,int> min_cost_max_flow(int st,int ed){
        memset(h,0,sizeof h);
        for(int flow=0,cost=0,res=inf;;res=inf){
            
            dijkstra(st);
            if(dis[ed]>inf)return {flow,cost};
            for(int i=0;i<N;i++){
                h[i]+=dis[i];
            }
            for(int i=ed;i!=st;i=preu[i]){
                res=min(res,g[preu[i]][pre[i]].cap);
            }
            flow+=res;
            cost+=res*h[ed];
            for(int i=ed;i!=st;i=preu[i]){
                g[i][g[preu[i]][pre[i]].inv].cap+=res;
                g[preu[i]][pre[i]].cap-=res;
            }
        }
    }
};
MCMF<505>mcmf;
int n,m,s,t;
signed main(){
    cin>>n>>m>>s>>t;
    for(int i=1;i<=m;i++){
        int u,v,w,c;cin>>u>>v>>w>>c;
        mcmf.add(u,v,w,c);
    }
    auto [f,c]=mcmf.min_cost_max_flow(s,t);
    cout<<f<<" "<<c<<endl;
}
\end{minted}

\hypertarget{scc_tarjanux6709ux5411ux56feux5f3aux8fdeux901aux5206ux91cf}{%
\subsection{9.Scc\_Tarjan(有向图强连通分量)}\label{scc_tarjanux6709ux5411ux56feux5f3aux8fdeux901aux5206ux91cf}}

\begin{minted}[fontsize=\footnotesize,breaklines,linenos]{cpp}
#include <bits/stdc++.h>
using namespace std;
int n,m;
int newid[200005];
vector<int>p[200005];
vector<vector<int>>scc;
int dfn[200005],low[200005],ins[200005],idx,cnt;
vector<int>st,f;
void scc_tarjan(int u){
    low[u]=dfn[u]=++idx;
    st.push_back(u);
    ins[u]=1;
    for(auto v:p[u]){
        if(!dfn[v])scc_tarjan(v);
        if(ins[v])low[u]=min(low[u],low[v]);
    }
    if(low[u]==dfn[u]){
        ++cnt;
        f.clear();
        while(1){
            int v=st.back();st.pop_back();
            f.push_back(v);
            ins[v]=0;
            newid[v]=cnt;
            if(u==v)break;
        }
        scc.push_back(f);
    }
}

signed main(){
    cin>>n>>m;
    for(int i=1;i<=m;i++){
        int u,v;cin>>u>>v;
        p[u].push_back(v);
    }
    for(int i=1;i<=n;i++){
        if(dfn[i])continue;
        scc_tarjan(i);
    }
}
\end{minted}

\hypertarget{edcc_tarjanux8fb9ux53ccux8fdeux901aux5206ux91cf}{%
\subsection{10.Edcc\_Tarjan(边双连通分量)}\label{edcc_tarjanux8fb9ux53ccux8fdeux901aux5206ux91cf}}

\begin{minted}[fontsize=\footnotesize,breaklines,linenos]{cpp}
#include <bits/stdc++.h>
using namespace std;
int n,m;
int newid[500005],cnt;
vector<int>g[200005];
vector<vector<int>>Edcc;
//下部分为不同题目所需变量
vector<int>ans;

//下为原图+跑tarjan所需变量
vector<pair<int,int>>p[500005];
int dfn[500005],low[500005],idx;
vector<int>st,f;
void Edcc_tarjan(int u,int id){
    low[u]=dfn[u]=++idx;
    st.push_back(u);
    for(auto [v,x]:p[u]){
        if(!dfn[v])Edcc_tarjan(v,x);
        if(id!=x)low[u]=min(low[u],low[v]);
    }
    if(id!=0&&low[u]==dfn[u]){
        f.push_back(id);
    }
    if(low[u]==dfn[u]){
        ++cnt;
        f.clear();
        while(1){
            int v=st.back();st.pop_back();
            f.push_back(v);
            newid[v]=cnt;
            if(u==v)break;
        }
        Edcc.push_back(f);
    }
}
signed main(){
    cin>>n>>m;
    for(int i=1;i<=m;i++){
        int u,v;cin>>u>>v;
        p[u].push_back({v,i});
        p[v].push_back({u,i});
    }
    for(int i=1;i<=n;i++){
        if(dfn[i])continue;
        Edcc_tarjan(i,0);
    }
    cout<<Edcc.size()<<endl;
    for(auto &x:Edcc){
        cout<<x.size()<<" ";
        for(auto &y:x){
            cout<<y<<" ";
        }
        cout<<endl;
    }
}
\end{minted}

\hypertarget{vdcc_tarjan}{%
\subsection{11.Vdcc\_Tarjan}\label{vdcc_tarjan}}

\begin{minted}[fontsize=\footnotesize,breaklines,linenos]{cpp}
#include <bits/stdc++.h>
using namespace std;
int n,m;
int newid[500005],cnt,cut[500005];
vector<int>g[200005];
vector<vector<int>>Vdcc;
//下部分为不同题目所需变量
vector<int>ans;
//下为原图+跑tarjan所需变量
vector<int>p[500005];
int dfn[500005],low[500005],idx;
vector<int>st,f;

void Vdcc_tarjan(int u,int fa){
    low[u]=dfn[u]=++idx;
    st.push_back(u);
    int tot=0;
    for(auto v:p[u]){
        if(v==fa)continue;
        if(!dfn[v]){
            Vdcc_tarjan(v,u);tot++;
            low[u]=min(low[v],low[u]);
            if(low[v]>=dfn[u]){
                f.clear();
                f.push_back(u);
                while(1){
                    int x=st.back();st.pop_back();
                    f.push_back(x);
                    if(x==v)break;
                }
                Vdcc.push_back(f);
            }
        }
        low[u]=min(low[u],dfn[v]);
    }
    if(fa==0&&tot==0){
        f.clear();f.push_back(u);Vdcc.push_back(f);
    }
}
signed main(){
    cin>>n>>m;
    for(int i=1;i<=m;i++){
        int u,v;cin>>u>>v;
        if(u==v)continue;
        p[u].push_back(v);
        p[v].push_back(u);
    }
    for(int i=1;i<=n;i++){
        if(dfn[i])continue;
        Vdcc_tarjan(i,0);
    }
    cout<<Vdcc.size()<<endl;
    for(auto &x:Vdcc){
        cout<<x.size()<<" ";
        for(auto &y:x){
            cout<<y<<" ";
        }
        cout<<endl;
    }
}
\end{minted}

\end{document}
