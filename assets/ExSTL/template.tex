\PassOptionsToPackage{unicode=true}{hyperref} % options for packages loaded elsewhere
\PassOptionsToPackage{hyphens}{url}
%
\documentclass[]{article}
\usepackage{lmodern}
\usepackage{amssymb,amsmath}

% settings
\usepackage{minted}
\usepackage{longtable}
\usepackage{booktabs}
%



\usepackage{ifxetex,ifluatex}
\usepackage{fixltx2e} % provides \textsubscript
\ifnum 0\ifxetex 1\fi\ifluatex 1\fi=0 % if pdftex
  \usepackage[T1]{fontenc}
  \usepackage[utf8]{inputenc}
  \usepackage{textcomp} % provides euro and other symbols
\else % if luatex or xelatex
  \usepackage{unicode-math}
  \defaultfontfeatures{Ligatures=TeX,Scale=MatchLowercase}
    \setmainfont[]{Source Han Serif CN}
    \setsansfont[]{Source Han Sans CN}
    \setmonofont[Mapping=tex-ansi]{Source Code Pro}
  \ifxetex
    \usepackage{xeCJK}
    \setCJKmainfont[]{Source Han Serif CN}
  \fi
  \ifluatex
    \usepackage[]{luatexja-fontspec}
    \setmainjfont[]{Source Han Serif CN}
  \fi
\fi
% use upquote if available, for straight quotes in verbatim environments
\IfFileExists{upquote.sty}{\usepackage{upquote}}{}
% use microtype if available
\IfFileExists{microtype.sty}{%
\usepackage[]{microtype}
\UseMicrotypeSet[protrusion]{basicmath} % disable protrusion for tt fonts
}{}
\IfFileExists{parskip.sty}{%
\usepackage{parskip}
}{% else
\setlength{\parindent}{0pt}
\setlength{\parskip}{6pt plus 2pt minus 1pt}
}
\usepackage{hyperref}
\hypersetup{
            pdfborder={0 0 0},
            breaklinks=true}
\urlstyle{same}  % don't use monospace font for urls
\usepackage[margin=2cm]{geometry}
\setlength{\emergencystretch}{3em}  % prevent overfull lines
\providecommand{\tightlist}{%
  \setlength{\itemsep}{0pt}\setlength{\parskip}{0pt}}
\setcounter{secnumdepth}{0}
% Redefines (sub)paragraphs to behave more like sections
\ifx\paragraph\undefined\else
\let\oldparagraph\paragraph
\renewcommand{\paragraph}[1]{\oldparagraph{#1}\mbox{}}
\fi
\ifx\subparagraph\undefined\else
\let\oldsubparagraph\subparagraph
\renewcommand{\subparagraph}[1]{\oldsubparagraph{#1}\mbox{}}
\fi


% set default figure placement to htbp
\makeatletter
\def\fps@figure{htbp}
\makeatother

\usepackage{minted}



\date{}

\title{\vspace{50mm} \huge Standard Code Library(ExSTL Part) \\[20pt]}
\author{F0RE1GNERS \\[10pt] Jiangxi Normal University HeartFireY}
\date{September 1202}


\begin{document}

\begin{titlepage}

\maketitle

\end{titlepage}

\newpage

\renewcommand\labelitemi{$\bullet$}

{
\setcounter{tocdepth}{3}
\tableofcontents
\newpage
}








\hypertarget{extcpbds-ux98dfux7528ux65b9ux6cd5}{%
\subsection{ExtC++(PBDS)
食用方法}\label{extcpbds-ux98dfux7528ux65b9ux6cd5}}

\hypertarget{part1.ux5f15ux5165}{%
\subsubsection{Part1.引入}\label{part1.ux5f15ux5165}}

pb\_ds 库全称 Policy-Based Data Structures。

pb\_ds
库封装了很多数据结构,比如哈希(Hash)表,平衡二叉树,字典树(Trie
树),堆(优先队列)等。

就像 \texttt{vector}、\texttt{set}、\texttt{map} 一样,其组件均符合 STL
的相关接口规范。部分(如优先队列)包含 STL 内对应组件的所有功能,但比
STL 功能更多。

pb\_ds 只在使用 libstdc++ 为标准库的编译器下可以用。

引入方法:

\begin{minted}[fontsize=\footnotesize,breaklines,linenos]{cpp}
#include <ext/pb_ds/assoc_container.hpp>
#include <ext/pb_ds/tree_policy.hpp>    // 引入平衡树
#include <ext/pb_ds/hash_policy.hpp>    // 引入hash
#include <ext/pb_ds/trie_policy.hpp>    // 引入trie
#include <ext/pb_ds/priority_queue.hpp> // 引入priority_queue
using namespace __gnu_pbds;
\end{minted}

更为简洁的引入方式:

\begin{minted}[fontsize=\footnotesize,breaklines,linenos]{cpp}
#include <bits/extc++.h>    //直接全部淦进来
using namespace __gnu_pbds;
\end{minted}

\hypertarget{part2.ux7528ux6cd5}{%
\subsubsection{Part2.用法}\label{part2.ux7528ux6cd5}}

\hypertarget{ux5e73ux8861ux6811tree}{%
\paragraph{(1).平衡树:Tree}\label{ux5e73ux8861ux6811tree}}

\hypertarget{ux8981ux6c42ux5f15ux5165ux5934ux6587ux4ef6}{%
\subparagraph{要求引入头文件:}\label{ux8981ux6c42ux5f15ux5165ux5934ux6587ux4ef6}}

\begin{minted}[fontsize=\footnotesize,breaklines,linenos]{cpp}
#include <ext/pb_ds/assoc_container.hpp>
#include <ext/pb_ds/tree_policy.hpp>
using namespace __gnu_pbds;
\end{minted}

\hypertarget{ux6784ux9020ux65b9ux5f0f}{%
\subparagraph{构造方式:}\label{ux6784ux9020ux65b9ux5f0f}}

\begin{minted}[fontsize=\footnotesize,breaklines,linenos]{cpp}
template <
    typename Key,
    typename Mapped,
    typename Cmp_Fn = std::less<Key>,
    typename Tag = rb_tree_tag,
    template<
        typename Const_Node_Iterator,
        typename Node_Iterator,
        typename Cmp_Fn_,
        typename Allocator_>
    class Node_Update = null_tree_node_update,
    typename Allocator = std::allocator<char>> class tree;
\end{minted}

\hypertarget{ux6a21ux677fux5f62ux53c2}{%
\subparagraph{模板形参:}\label{ux6a21ux677fux5f62ux53c2}}

\begin{itemize}
\tightlist
\item
  \texttt{Key}: 储存的元素类型,如果想要存储多个相同的 \texttt{Key}
  元素,则需要使用类似于 \texttt{std::pair} 和 \texttt{struct}
  的方法,并配合使用 \texttt{lower\_bound} 和 \texttt{upper\_bound}
  成员函数进行查找
\item
  \texttt{Mapped}: 映射规则(Mapped-Policy)类型,如果要指示关联容器是
  \textbf{集合},类似于存储元素在 \texttt{std::set} 中,此处填入
  \texttt{null\_type},低版本 \texttt{g++} 此处为
  \texttt{null\_mapped\_type};如果要指示关联容器是
  \textbf{带值的集合},类似于存储元素在 \texttt{std::map}
  中,此处填入类似于
  \texttt{std::map\textless{}Key,\ Value\textgreater{}} 的
  \texttt{Value} 类型
\item
  \texttt{Cmp\_Fn}: 关键字比较函子,例如
  \texttt{std::less\textless{}Key\textgreater{}}
\item
  \texttt{Tag}: 选择使用何种底层数据结构类型,默认是
  \texttt{rb\_tree\_tag}。\texttt{\_\_gnu\_pbds}
  提供不同的三种平衡树,分别是:

  \begin{itemize}
  \tightlist
  \item
    \texttt{rb\_tree\_tag}:红黑树,{[}\textbf{一般使用这个}{]},后两者的性能一般不如红黑树,容易被卡
  \item
    \texttt{splay\_tree\_tag}:splay 树
  \item
    \texttt{ov\_tree\_tag}:有序向量树,只是一个由 \texttt{vector}
    实现的有序结构,类似于排序的 \texttt{vector}
    来实现平衡树,性能取决于数据想不想卡你
  \end{itemize}
\item
  \texttt{Node\_Update}:用于更新节点的策略,默认使用
  \texttt{null\_node\_update},若要使用 \texttt{order\_of\_key} 和
  \texttt{find\_by\_order} 方法,需要使用
  \texttt{tree\_order\_statistics\_node\_update}(该方法是在统计子树的\(size\))
\item
  \texttt{Allocator}:空间分配器类型
\end{itemize}

\hypertarget{ux6210ux5458ux51fdux6570}{%
\subparagraph{成员函数:}\label{ux6210ux5458ux51fdux6570}}

\begin{itemize}
\tightlist
\item
  \texttt{insert(x)}:向树中插入一个元素 x,返回
  \texttt{std::pair\textless{}point\_iterator,\ bool\textgreater{}}。
\item
  \texttt{erase(x)}:从树中删除一个元素/迭代器 x,返回一个 \texttt{bool}
  表明是否删除成功。
\item
  \texttt{order\_of\_key(x)}:返回 x 以 \texttt{Cmp\_Fn} 比较的排名。
\item
  \texttt{find\_by\_order(x)}:返回 \texttt{Cmp\_Fn}
  比较的排名所对应元素的迭代器。
\item
  \texttt{lower\_bound(x)}:以 \texttt{Cmp\_Fn} 比较做
  \texttt{lower\_bound},返回迭代器。
\item
  \texttt{upper\_bound(x)}:以 \texttt{Cmp\_Fn} 比较做
  \texttt{upper\_bound},返回迭代器。
\item
  \texttt{join(x)}:将 x 树并入当前树,前提是两棵树的类型一样,x
  树被删除。
\item
  \texttt{split(x,b)}:以 \texttt{Cmp\_Fn} 比较,小于等于 x
  的属于当前树,其余的属于 b 树。
\item
  \texttt{empty()}:返回是否为空。
\item
  \texttt{size()}:返回大小。
\end{itemize}

\hypertarget{ux81eaux4e49ux5b9anode_update}{%
\subparagraph{自义定Node\_update}\label{ux81eaux4e49ux5b9anode_update}}

\begin{minted}[fontsize=\footnotesize,breaklines,linenos]{cpp}
template <class Node_CItr, class Node_Itr, class Cmp_Fn, class _Alloc>
struct my_node_update { 
    virtual Node_CItr node_begin() const = 0;
    virtual Node_CItr node_end() const = 0;
    typedef int metadata_type; // metadata type: 是指节点上记录的额外信息的类型
    // operator() 的功能是将节点it的信息更新为其左右儿子的信息之和,传入的end_it表示空节点
    // it 是Node_Iter, 用星号进行取值后变为iterator, -> second即为该节点的mapped_value
    inline void operator()(Node_Itr it, Node_CItr end_it) {
        Node_Itr l = it.get_l_child(), r = it.get_r_child();
        int left = 0, right = 0;
        if(l != end_it) left = l.get_metadata();
        if(r != end_it) right = r.get_metadata();
        const_cast<metadata_type &>(it.get_metadata()) = left + right + 1;
    }
    inline int order_of_key(pair<int, int> x) {
        int ans = 0;
        Node_CItr it = node_begin();
        while(it != node_end()) {
            Node_CItr l = it.get_l_child();
            Node_CItr r = it.get_r_child();
            if(Cmp_Fn()(x, **it)) it = l;
            else {
                ans++;
                if(l != node_end()) ans += l.get_metadata();
                it = r;
            }
        }
        return ans;
    }
};

tree<pair<int, int>, null_type, less<pair<int, int>>, rb_tree_tag, my_node_update> tr;
\end{minted}

\hypertarget{ux5b57ux5178ux6811-trie}{%
\paragraph{(2).字典树 Trie}\label{ux5b57ux5178ux6811-trie}}

\hypertarget{ux8981ux6c42ux5f15ux5165ux5934ux6587ux4ef6-1}{%
\subparagraph{要求引入头文件}\label{ux8981ux6c42ux5f15ux5165ux5934ux6587ux4ef6-1}}

\begin{minted}[fontsize=\footnotesize,breaklines,linenos]{cpp}
#include <ext/pb_ds/assoc_container.hpp>
#include <ext/pb_ds/trie_policy.hpp>
using namespace __gnu_pbds;
\end{minted}

\hypertarget{ux6784ux9020ux65b9ux5f0fux4ee5ux53caux4f7fux7528ux65b9ux6cd5}{%
\subparagraph{构造方式以及使用方法}\label{ux6784ux9020ux65b9ux5f0fux4ee5ux53caux4f7fux7528ux65b9ux6cd5}}

\begin{minted}[fontsize=\footnotesize,breaklines,linenos]{cpp}
typedef trie<string,null_type,trie_string_access_traits<>,pat_trie_tag,trie_prefix_search_node_update> tr;
//第一个参数必须为字符串类型,tag也有别的tag,但pat最快,与tree相同,node_update支持自定义
tr.insert(s); //插入s 
tr.erase(s); //删除s 
tr.join(b); //将b并入tr 
pair//pair的使用如下:
pair<tr::iterator,tr::iterator> range=base.prefix_range(x);
for(tr::iterator it=range.first;it!=range.second;it++) cout<<*it<<' '<<endl;
//pair中第一个是起始迭代器,第二个是终止迭代器,遍历过去就可以找到所有字符串了。 
\end{minted}

\hypertarget{ux54c8ux5e0cux8868-hashtable}{%
\paragraph{(3).哈希表 HashTable}\label{ux54c8ux5e0cux8868-hashtable}}

\hypertarget{ux8981ux6c42ux5f15ux5165ux5934ux6587ux4ef6-2}{%
\subparagraph{要求引入头文件}\label{ux8981ux6c42ux5f15ux5165ux5934ux6587ux4ef6-2}}

\begin{minted}[fontsize=\footnotesize,breaklines,linenos]{cpp}
#include <ext/pb_ds/assoc_container.hpp>
#include <ext/pb_ds/hash_policy.hpp>    // 引入hash
using namespace __gnu_pbds;
\end{minted}

\hypertarget{ux4f7fux7528ux65b9ux6cd5}{%
\subparagraph{使用方法}\label{ux4f7fux7528ux65b9ux6cd5}}

\begin{minted}[fontsize=\footnotesize,breaklines,linenos]{cpp}
cc_hash_table<int, bool> h;  // 拉链法
gp_hash_table<int, bool> h;  // 探测法(推荐)
\end{minted}

其余方法同\texttt{std::map},但是注意,该数据结构的总复杂度是\(O(N)\)。

\hypertarget{ux5806-priority_queue}{%
\paragraph{(4).堆 Priority\_queue}\label{ux5806-priority_queue}}

附:\href{https://gcc.gnu.org/onlinedocs/libstdc++/ext/pb_ds/pq_performance_tests.html\#std_mod1}{官方文档地址------复杂度及常数测试}

\begin{minted}[fontsize=\footnotesize,breaklines,linenos]{cpp}
#include <ext/pb_ds/priority_queue.hpp>
using namespace __gnu_pbds;
__gnu_pbds ::priority_queue<T, Compare, Tag, Allocator>
\end{minted}

\hypertarget{ux6a21ux677fux5f62ux53c2-1}{%
\subparagraph{模板形参}\label{ux6a21ux677fux5f62ux53c2-1}}

\begin{itemize}
\tightlist
\item
  \texttt{T}: 储存的元素类型
\item
  \texttt{Compare}: 提供严格的弱序比较类型
\item
  \texttt{Tag}: 是 \texttt{\_\_gnu\_pbds} 提供的不同的五种堆,Tag
  参数默认是 \texttt{pairing\_heap\_tag} 五种分别是:
\item
  \texttt{pairing\_heap\_tag}:配对堆
  官方文档认为在非原生元素(如自定义结构体/\texttt{std\ ::\ string}/\texttt{pair})
  中,配对堆表现最好
\item
  \texttt{binary\_heap\_tag}:二叉堆
  官方文档认为在原生元素中二叉堆表现最好,不过我测试的表现并没有那么好
\item
  \texttt{binomial\_heap\_tag}:二项堆
  二项堆在合并操作的表现要优于二叉堆,但是其取堆顶元素操作的复杂度比二叉堆高
\item
  \texttt{rc\_binomial\_heap\_tag}:冗余计数二项堆
\item
  \texttt{thin\_heap\_tag}:除了合并的复杂度都和 Fibonacci 堆一样的一个
  tag
\item
  \texttt{Allocator}:空间配置器,由于 OI 中很少出现,故这里不做讲解
\end{itemize}

由于本篇文章只是提供给学习算法竞赛的同学们,故对于后四个 tag
只会简单的介绍复杂度,第一个会介绍成员函数和使用方法。

经作者本机 Core i5 @3.1 GHz On macOS 测试堆的基础操作,结合 GNU
官方的复杂度测试,Dijkstra 测试,都表明: 至少对于 OIer
来讲,除了配对堆的其他四个 tag 都是鸡肋,要么没用,要么常数大到不如
\texttt{std} 的,且有可能造成
MLE,故这里只推荐用默认的配对堆。同样,配对堆也优于 \texttt{algorithm}
库中的 \texttt{make\_heap()}。

\hypertarget{ux6784ux9020ux65b9ux5f0f-1}{%
\subparagraph{构造方式}\label{ux6784ux9020ux65b9ux5f0f-1}}

要注明命名空间因为和 \texttt{std} 的类名称重复。

\begin{verbatim}
__gnu_pbds ::priority_queue<int> __gnu_pbds::priority_queue<int, greater<int> >
__gnu_pbds ::priority_queue<int, greater<int>, pairing_heap_tag>
__gnu_pbds ::priority_queue<int>::point_iterator id; // 点类型迭代器
// 在 modify 和 push 的时候都会返回一个 point_iterator,下文会详细的讲使用方法
id = q.push(1);
\end{verbatim}

\hypertarget{ux6210ux5458ux51fdux6570-1}{%
\subparagraph{成员函数}\label{ux6210ux5458ux51fdux6570-1}}

\begin{itemize}
\tightlist
\item
  \texttt{push()}: 向堆中压入一个元素,返回该元素位置的迭代器。
\item
  \texttt{pop()}: 将堆顶元素弹出。
\item
  \texttt{top()}: 返回堆顶元素。
\item
  \texttt{size()} 返回元素个数。
\item
  \texttt{empty()} 返回是否非空。
\item
  \texttt{modify(point\_iterator,\ const\ key)}: 把迭代器位置的
  \texttt{key} 修改为传入的 \texttt{key},并对底层储存结构进行排序。
\item
  \texttt{erase(point\_iterator)}: 把迭代器位置的键值从堆中擦除。
\item
  \texttt{join(\_\_gnu\_pbds\ ::\ priority\_queue\ \&other)}: 把
  \texttt{other} 合并到 \texttt{*this} 并把 \texttt{other} 清空。
\end{itemize}

使用的 tag 决定了每个操作的时间复杂度:

\begin{longtable}[]{@{}lp{2cm}p{2cm}p{2cm}p{2cm}p{2cm}@{}}
\toprule()
& push & pop & modify & erase & Join \\
\midrule()
\endhead
\texttt{pairing\_heap\_tag} & \(O(1)\) & 最坏 \(\Theta(n)\)  均摊
\(\Theta(\log(n))\) & 最坏 \(\Theta(n)\) 均摊 \(\Theta(\log(n))\) & 最坏
\(\Theta(n)\) 均摊 \(\Theta(\log(n))\) & \(O(1)\) \\ \midrule
\texttt{binary\_heap\_tag} & 最坏 \(\Theta(n)\) 均摊 \(\Theta(\log(n))\)
& 最坏 \(\Theta(n)\) 均摊 \(\Theta(\log(n))\) & \(\Theta(n)\) &
\(\Theta(n)\) & \(\Theta(n)\) \\ \midrule
\texttt{binomial\_heap\_tag} & 最坏 \(\Theta(\log(n))\) 均摊 \(O(1)\) &
\(\Theta(\log(n))\) & \(\Theta(\log(n))\) & \(\Theta(\log(n))\) &
\(\Theta(\log(n))\) \\\midrule
\texttt{rc\_binomial\_heap\_tag} & \(O(1)\) & \(\Theta(\log(n))\) &
\(\Theta(\log(n))\) & \(\Theta(\log(n))\) & \(\Theta(\log(n))\) \\ \midrule
\texttt{thin\_heap\_tag} & \(O(1)\) & 最坏 \(\Theta(n)\) 均摊
\(\Theta(\log(n))\) & 最坏 \(\Theta(\log(n))\) 均摊 \(O(1)\) & 最坏
\(\Theta(n)\) 0 均摊 \(\Theta(\log(n))\) & \(\Theta(n)\) \\
\bottomrule()
\end{longtable}

\begin{minted}[fontsize=\footnotesize,breaklines,linenos]{cpp}
#include <algorithm>
#include <cstdio>
#include <ext/pb_ds/priority_queue.hpp>
#include <iostream>
using namespace __gnu_pbds;
// 由于面向OIer, 本文以常用堆 : pairing_heap_tag作为范例
// 为了更好的阅读体验,定义宏如下 :
#define pair_heap __gnu_pbds ::priority_queue<int>
pair_heap q1;  // 大根堆, 配对堆
pair_heap q2;
pair_heap ::point_iterator id;  // 一个迭代器

int main() {
  id = q1.push(1);
  // 堆中元素 : [1];
  for (int i = 2; i <= 5; i++) q1.push(i);
  // 堆中元素 :  [1, 2, 3, 4, 5];
  std ::cout << q1.top() << std ::endl;
  // 输出结果 : 5;
  q1.pop();
  // 堆中元素 : [1, 2, 3, 4];
  id = q1.push(10);
  // 堆中元素 : [1, 2, 3, 4, 10];
  q1.modify(id, 1);
  // 堆中元素 :  [1, 1, 2, 3, 4];
  std ::cout << q1.top() << std ::endl;
  // 输出结果 : 4;
  q1.pop();
  // 堆中元素 : [1, 1, 2, 3];
  id = q1.push(7);
  // 堆中元素 : [1, 1, 2, 3, 7];
  q1.erase(id);
  // 堆中元素 : [1, 1, 2, 3];
  q2.push(1), q2.push(3), q2.push(5);
  // q1中元素 : [1, 1, 2, 3], q2中元素 : [1, 3, 5];
  q2.join(q1);
  // q1中无元素,q2中元素 :[1, 1, 1, 2, 3, 3, 5];
}
\end{minted}

\hypertarget{gnu_pbds-ux8fedux4ee3ux5668ux7684ux5931ux6548ux4fddux8bc1invalidation_guarantee}{%
\subparagraph{\_\_gnu\_pbds
迭代器的失效保证(invalidation\_guarantee)}\label{gnu_pbds-ux8fedux4ee3ux5668ux7684ux5931ux6548ux4fddux8bc1invalidation_guarantee}}

在上述示例以及一些实践中(如使用本章的 pb-ds
堆来编写单源最短路等算法),常常需要保存并使用堆的迭代器(如
\texttt{\_\_gnu\_pbds::priority\_queue\textless{}int\textgreater{}::point\_iterator}
等)。

可是例如对于 \texttt{\_\_gnu\_pbds::priority\_queue} 中不同的 Tag
参数,其底层实现并不相同,迭代器的失效条件也不一样,根据\_\_gnu\_pbds
库的设计,以下三种由上至下派生的情况:

\begin{enumerate}
\def\labelenumi{\arabic{enumi}.}
\item
  基本失效保证(basic\_invalidation\_guarantee):即不修改容器时,点类型迭代器(point\_iterator)、指针和引用(key/value)\textbf{保持}
  有效。
\item
  点失效保证(point\_invalidation\_guarantee):即 \textbf{修改}
  容器后,点类型迭代器(point\_iterator)、指针和引用(key/value)只要对应在容器中没被删除
  \textbf{保持} 有效。
\item
  范围失效保证(range\_invalidation\_guarantee):即 \textbf{修改}
  容器后,除(2)的特性以外,任何范围类型的迭代器(包括 \texttt{begin()}
  和 \texttt{end()} 的返回值)是正确的,具有范围失效保证的 Tag 有
  rb\_tree\_tag 和 适用于 \texttt{\_\_gnu\_pbds::tree} 的
  splay\_tree\_tag(),以及 适用于 \texttt{\_\_gnu\_pbds::trie} 的
  pat\_trie\_tag。
\end{enumerate}

从运行下述代码中看出,除了 \texttt{binary\_heap\_tag} 为
\texttt{basic\_invalidation\_guarantee} 在修改后迭代器会失效,其余的均为
\texttt{point\_invalidation\_guarantee} 可以实现修改后点类型迭代器
(point\_iterator) 不失效的需求。

\begin{minted}[fontsize=\footnotesize,breaklines,linenos]{cpp}
#include <bits/stdc++.h>
using namespace std;
#include <ext/pb_ds/assoc_container.hpp>
#include <ext/pb_ds/priority_queue.hpp>
using namespace __gnu_pbds;
#include <cxxabi.h>

template <typename T>
void print_invalidation_guarantee() {
  typedef typename __gnu_pbds::container_traits<T>::invalidation_guarantee gute;
  cout << abi::__cxa_demangle(typeid(gute).name(), 0, 0, 0) << endl;
}

int main() {
  typedef
      typename __gnu_pbds::priority_queue<int, greater<int>, pairing_heap_tag>
          pairing;
  typedef
      typename __gnu_pbds::priority_queue<int, greater<int>, binary_heap_tag>
          binary;
  typedef
      typename __gnu_pbds::priority_queue<int, greater<int>, binomial_heap_tag>
          binomial;
  typedef typename __gnu_pbds::priority_queue<int, greater<int>,
                                              rc_binomial_heap_tag>
      rc_binomial;
  typedef typename __gnu_pbds::priority_queue<int, greater<int>, thin_heap_tag>
      thin;
  print_invalidation_guarantee<pairing>();
  print_invalidation_guarantee<binary>();
  print_invalidation_guarantee<binomial>();
  print_invalidation_guarantee<rc_binomial>();
  print_invalidation_guarantee<thin>();
  return 0;
}
\end{minted}

\hypertarget{ux53efux6301ux4e45ux5316ux6570ux7ec4ux53efux6301ux4e45ux5316ux5e73ux8861ux6811ux5757ux72b6ux94feux8868-rope}{%
\paragraph{(5).可持久化数组/可持久化平衡树/块状链表
rope}\label{ux53efux6301ux4e45ux5316ux6570ux7ec4ux53efux6301ux4e45ux5316ux5e73ux8861ux6811ux5757ux72b6ux94feux8868-rope}}

\hypertarget{ux8981ux6c42ux5f15ux5165ux5934ux6587ux4ef6-3}{%
\subparagraph{要求引入头文件}\label{ux8981ux6c42ux5f15ux5165ux5934ux6587ux4ef6-3}}

\begin{minted}[fontsize=\footnotesize,breaklines,linenos]{cpp}
#include <ext/rope>
using namespace __gnu_cxx;
\end{minted}

\hypertarget{ux4f7fux7528ux65b9ux6cd5-1}{%
\subparagraph{使用方法}\label{ux4f7fux7528ux65b9ux6cd5-1}}

\begin{minted}[fontsize=\footnotesize,breaklines,linenos]{cpp}
// 定义:
rope<int> rp;
\end{minted}

\hypertarget{ux6210ux5458ux51fdux6570-2}{%
\subparagraph{成员函数}\label{ux6210ux5458ux51fdux6570-2}}

\begin{itemize}
\item
  \texttt{push\_back(x)}: 在末尾插入\(x\)
\item
  \texttt{insert(pos,\ x)}: 在\(pos\)处插入\(x\)
\item
  \texttt{erase(pos,\ x)}: 在\(pos\)处删除\(x\)个元素
\item
  \texttt{length()}: 返回数组长度
\item
  \texttt{size()}: 返回数组长度(同上)
\item
  \texttt{replace(pos,\ x)}: 将\(pos\)处元素替换为\(x\)
\item
  \texttt{substr(pos,\ x,\ s)}: 从\(pos\)处开始提取\(x\)个元素
\item
  \texttt{copy(pos,\ x,\ s)}: 从\(pos\)处开始复制\(x\)个元素到\(s\)中
\item
  \texttt{at(x)}: 访问第\(x\)个元素,同\texttt{rp{[}x{]}}
\end{itemize}

\texttt{rope} 内部是块状链表实现的,黑科技是支持 \(O(1)\)
复制,而且不会空间爆炸 (\texttt{rope}
是平衡树,拷贝时只拷贝根节点就行)。因此可以用来做可持久化数组。

拷贝历史版本的方式:

\begin{minted}[fontsize=\footnotesize,breaklines,linenos]{cpp}
rope<int> *his[100000];
his[i] = new rope<int> (*his[i - 1]);
\end{minted}

\end{document}
